\documentclass[12pt,a4paper,]{book}
\def\ifdoblecara{} %% set to true
\def\ifprincipal{} %% set to true
\let\ifprincipal\undefined %% set to false
\def\ifcitapandoc{} %% set to true
\let\ifcitapandoc\undefined %% set to false
\usepackage{lmodern}
% sin fontmathfamily
\usepackage{amssymb,amsmath}
\usepackage{ifxetex,ifluatex}
%\usepackage{fixltx2e} % provides \textsubscript %PLLC
\ifnum 0\ifxetex 1\fi\ifluatex 1\fi=0 % if pdftex
  \usepackage[T1]{fontenc}
  \usepackage[utf8]{inputenc}
\else % if luatex or xelatex
  \ifxetex
    \usepackage{mathspec}
  \else
    \usepackage{fontspec}
  \fi
  \defaultfontfeatures{Ligatures=TeX,Scale=MatchLowercase}
\fi
% use upquote if available, for straight quotes in verbatim environments
\IfFileExists{upquote.sty}{\usepackage{upquote}}{}
% use microtype if available
\IfFileExists{microtype.sty}{%
\usepackage{microtype}
\UseMicrotypeSet[protrusion]{basicmath} % disable protrusion for tt fonts
}{}
\usepackage[margin = 2.5cm]{geometry}
\usepackage{hyperref}
\hypersetup{unicode=true,
            pdfauthor={Nombre Completo Autor},
              pdfborder={0 0 0},
              breaklinks=true}
\urlstyle{same}  % don't use monospace font for urls
%
\usepackage[usenames,dvipsnames]{xcolor}  %new PLLC
\IfFileExists{parskip.sty}{%
\usepackage{parskip}
}{% else
\setlength{\parindent}{0pt}
\setlength{\parskip}{6pt plus 2pt minus 1pt}
}
\setlength{\emergencystretch}{3em}  % prevent overfull lines
\providecommand{\tightlist}{%
  \setlength{\itemsep}{0pt}\setlength{\parskip}{0pt}}
\setcounter{secnumdepth}{5}
% Redefines (sub)paragraphs to behave more like sections
\ifx\paragraph\undefined\else
\let\oldparagraph\paragraph
\renewcommand{\paragraph}[1]{\oldparagraph{#1}\mbox{}}
\fi
\ifx\subparagraph\undefined\else
\let\oldsubparagraph\subparagraph
\renewcommand{\subparagraph}[1]{\oldsubparagraph{#1}\mbox{}}
\fi

%%% Use protect on footnotes to avoid problems with footnotes in titles
\let\rmarkdownfootnote\footnote%
\def\footnote{\protect\rmarkdownfootnote}


  \title{}
    \author{Nombre Completo Autor}
      \date{18/11/2021}


%%%%%%% inicio: latex_preambulo.tex PLLC


%% UTILIZA CODIFICACIÓN UTF-8
%% MODIFICARLO CONVENIENTEMENTE PARA USARLO CON OTRAS CODIFICACIONES


%\usepackage[spanish,es-nodecimaldot,es-noshorthands]{babel}
\usepackage[spanish,es-nodecimaldot,es-noshorthands,es-tabla]{babel}
% Ver: es-tabla (en: https://osl.ugr.es/CTAN/macros/latex/contrib/babel-contrib/spanish/spanish.pdf)
% es-tabla (en: https://tex.stackexchange.com/questions/80443/change-the-word-table-in-table-captions)
\usepackage[spanish, plain, datebegin,sortcompress,nocomment,
noabstract]{flexbib}
 
\usepackage{float}
\usepackage{placeins}
\usepackage{fancyhdr}
% Solucion: ! LaTeX Error: Command \counterwithout already defined.
% https://tex.stackexchange.com/questions/425600/latex-error-command-counterwithout-already-defined
\let\counterwithout\relax
\let\counterwithin\relax
\usepackage{chngcntr}
%\usepackage{microtype}  %antes en template PLLC
\usepackage[utf8]{inputenc}
\usepackage[T1]{fontenc} % Usa codificación 8-bit que tiene 256 glyphs

%\usepackage[dvipsnames]{xcolor}
%\usepackage[usenames,dvipsnames]{xcolor}  %new
\usepackage{pdfpages}
%\usepackage{natbib}




% Para portada: latex_paginatitulo_mod_ST02.tex (inicio)
\usepackage{tikz}
\usepackage{epigraph}
\input{portadas/latex_paginatitulo_mod_ST02_add.sty}
% Para portada: latex_paginatitulo_mod_ST02.tex (fin)

% Para portada: latex_paginatitulo_mod_OV01.tex (inicio)
\usepackage{cpimod}
% Para portada: latex_paginatitulo_mod_OV01.tex (fin)

% Para portada: latex_paginatitulo_mod_OV03.tex (inicio)
\usepackage{KTHEEtitlepage}
% Para portada: latex_paginatitulo_mod_OV03.tex (fin)

\renewcommand{\contentsname}{Índice}
\renewcommand{\listfigurename}{Índice de figuras}
\renewcommand{\listtablename}{Índice de tablas}
\newcommand{\bcols}{}
\newcommand{\ecols}{}
\newcommand{\bcol}[1]{\begin{minipage}{#1\linewidth}}
\newcommand{\ecol}{\end{minipage}}
\newcommand{\balertblock}[1]{\begin{alertblock}{#1}}
\newcommand{\ealertblock}{\end{alertblock}}
\newcommand{\bitemize}{\begin{itemize}}
\newcommand{\eitemize}{\end{itemize}}
\newcommand{\benumerate}{\begin{enumerate}}
\newcommand{\eenumerate}{\end{enumerate}}
\newcommand{\saltopagina}{\newpage}
\newcommand{\bcenter}{\begin{center}}
\newcommand{\ecenter}{\end{center}}
\newcommand{\beproof}{\begin{proof}} %new
\newcommand{\eeproof}{\end{proof}} %new
%De: https://texblog.org/2007/11/07/headerfooter-in-latex-with-fancyhdr/
% \fancyhead
% E: Even page
% O: Odd page
% L: Left field
% C: Center field
% R: Right field
% H: Header
% F: Footer
%\fancyhead[CO,CE]{Resultados}

%OPCION 1
% \fancyhead[LE,RO]{\slshape \rightmark}
% \fancyhead[LO,RE]{\slshape \leftmark}
% \fancyfoot[C]{\thepage}
% \renewcommand{\headrulewidth}{0.4pt}
% \renewcommand{\footrulewidth}{0pt}

%OPCION 2
% \fancyhead[LE,RO]{\slshape \rightmark}
% \fancyfoot[LO,RE]{\slshape \leftmark}
% \fancyfoot[LE,RO]{\thepage}
% \renewcommand{\headrulewidth}{0.4pt}
% \renewcommand{\footrulewidth}{0.4pt}
%%%%%%%%%%
\usepackage{calc,amsfonts}
% Elimina la cabecera de páginas impares vacías al finalizar los capítulos
\usepackage{emptypage}
\makeatletter

%\definecolor{ocre}{RGB}{25,25,243} % Define el color azul (naranja) usado para resaltar algunas salidas
\definecolor{ocre}{RGB}{0,0,0} % Define el color a negro (aparece en los teoremas

%\usepackage{calc} 


%era if(csl-refs) con dolares
% metodobib: true


\usepackage{lipsum}

%\usepackage{tikz} % Requerido para dibujar formas personalizadas

%\usepackage{amsmath,amsthm,amssymb,amsfonts}
\usepackage{amsthm}


% Boxed/framed environments
\newtheoremstyle{ocrenumbox}% % Theorem style name
{0pt}% Space above
{0pt}% Space below
{\normalfont}% % Body font
{}% Indent amount
{\small\bf\sffamily\color{ocre}}% % Theorem head font
{\;}% Punctuation after theorem head
{0.25em}% Space after theorem head
{\small\sffamily\color{ocre}\thmname{#1}\nobreakspace\thmnumber{\@ifnotempty{#1}{}\@upn{#2}}% Theorem text (e.g. Theorem 2.1)
\thmnote{\nobreakspace\the\thm@notefont\sffamily\bfseries\color{black}---\nobreakspace#3.}} % Optional theorem note
\renewcommand{\qedsymbol}{$\blacksquare$}% Optional qed square

\newtheoremstyle{blacknumex}% Theorem style name
{5pt}% Space above
{5pt}% Space below
{\normalfont}% Body font
{} % Indent amount
{\small\bf\sffamily}% Theorem head font
{\;}% Punctuation after theorem head
{0.25em}% Space after theorem head
{\small\sffamily{\tiny\ensuremath{\blacksquare}}\nobreakspace\thmname{#1}\nobreakspace\thmnumber{\@ifnotempty{#1}{}\@upn{#2}}% Theorem text (e.g. Theorem 2.1)
\thmnote{\nobreakspace\the\thm@notefont\sffamily\bfseries---\nobreakspace#3.}}% Optional theorem note

\newtheoremstyle{blacknumbox} % Theorem style name
{0pt}% Space above
{0pt}% Space below
{\normalfont}% Body font
{}% Indent amount
{\small\bf\sffamily}% Theorem head font
{\;}% Punctuation after theorem head
{0.25em}% Space after theorem head
{\small\sffamily\thmname{#1}\nobreakspace\thmnumber{\@ifnotempty{#1}{}\@upn{#2}}% Theorem text (e.g. Theorem 2.1)
\thmnote{\nobreakspace\the\thm@notefont\sffamily\bfseries---\nobreakspace#3.}}% Optional theorem note

% Non-boxed/non-framed environments
\newtheoremstyle{ocrenum}% % Theorem style name
{5pt}% Space above
{5pt}% Space below
{\normalfont}% % Body font
{}% Indent amount
{\small\bf\sffamily\color{ocre}}% % Theorem head font
{\;}% Punctuation after theorem head
{0.25em}% Space after theorem head
{\small\sffamily\color{ocre}\thmname{#1}\nobreakspace\thmnumber{\@ifnotempty{#1}{}\@upn{#2}}% Theorem text (e.g. Theorem 2.1)
\thmnote{\nobreakspace\the\thm@notefont\sffamily\bfseries\color{black}---\nobreakspace#3.}} % Optional theorem note
\renewcommand{\qedsymbol}{$\blacksquare$}% Optional qed square
\makeatother



% Define el estilo texto theorem para cada tipo definido anteriormente
\newcounter{dummy} 
\numberwithin{dummy}{section}
\theoremstyle{ocrenumbox}
\newtheorem{theoremeT}[dummy]{Teorema}  % (Pedro: Theorem)
\newtheorem{problem}{Problema}[chapter]  % (Pedro: Problem)
\newtheorem{exerciseT}{Ejercicio}[chapter] % (Pedro: Exercise)
\theoremstyle{blacknumex}
\newtheorem{exampleT}{Ejemplo}[chapter] % (Pedro: Example)
\theoremstyle{blacknumbox}
\newtheorem{vocabulary}{Vocabulario}[chapter]  % (Pedro: Vocabulary)
\newtheorem{definitionT}{Definición}[section]  % (Pedro: Definition)
\newtheorem{corollaryT}[dummy]{Corolario}  % (Pedro: Corollary)
\theoremstyle{ocrenum}
\newtheorem{proposition}[dummy]{Proposición} % (Pedro: Proposition)


\usepackage[framemethod=default]{mdframed}



\newcommand{\intoo}[2]{\mathopen{]}#1\,;#2\mathclose{[}}
\newcommand{\ud}{\mathop{\mathrm{{}d}}\mathopen{}}
\newcommand{\intff}[2]{\mathopen{[}#1\,;#2\mathclose{]}}
\newtheorem{notation}{Notation}[chapter]


\mdfdefinestyle{exampledefault}{%
rightline=true,innerleftmargin=10,innerrightmargin=10,
frametitlerule=true,frametitlerulecolor=green,
frametitlebackgroundcolor=yellow,
frametitlerulewidth=2pt}


% Theorem box
\newmdenv[skipabove=7pt,
skipbelow=7pt,
backgroundcolor=black!5,
linecolor=ocre,
innerleftmargin=5pt,
innerrightmargin=5pt,
innertopmargin=10pt,%5pt
leftmargin=0cm,
rightmargin=0cm,
innerbottommargin=5pt]{tBox}

% Exercise box	  
\newmdenv[skipabove=7pt,
skipbelow=7pt,
rightline=false,
leftline=true,
topline=false,
bottomline=false,
backgroundcolor=ocre!10,
linecolor=ocre,
innerleftmargin=5pt,
innerrightmargin=5pt,
innertopmargin=10pt,%5pt
innerbottommargin=5pt,
leftmargin=0cm,
rightmargin=0cm,
linewidth=4pt]{eBox}	

% Definition box
\newmdenv[skipabove=7pt,
skipbelow=7pt,
rightline=false,
leftline=true,
topline=false,
bottomline=false,
linecolor=ocre,
innerleftmargin=5pt,
innerrightmargin=5pt,
innertopmargin=10pt,%0pt
leftmargin=0cm,
rightmargin=0cm,
linewidth=4pt,
innerbottommargin=0pt]{dBox}	

% Corollary box
\newmdenv[skipabove=7pt,
skipbelow=7pt,
rightline=false,
leftline=true,
topline=false,
bottomline=false,
linecolor=gray,
backgroundcolor=black!5,
innerleftmargin=5pt,
innerrightmargin=5pt,
innertopmargin=10pt,%5pt
leftmargin=0cm,
rightmargin=0cm,
linewidth=4pt,
innerbottommargin=5pt]{cBox}

% Crea un entorno para cada tipo de theorem y le asigna un estilo 
% con ayuda de las cajas coloreadas anteriores
\newenvironment{theorem}{\begin{tBox}\begin{theoremeT}}{\end{theoremeT}\end{tBox}}
\newenvironment{exercise}{\begin{eBox}\begin{exerciseT}}{\hfill{\color{ocre}\tiny\ensuremath{\blacksquare}}\end{exerciseT}\end{eBox}}				  
\newenvironment{definition}{\begin{dBox}\begin{definitionT}}{\end{definitionT}\end{dBox}}	
\newenvironment{example}{\begin{exampleT}}{\hfill{\tiny\ensuremath{\blacksquare}}\end{exampleT}}		
\newenvironment{corollary}{\begin{cBox}\begin{corollaryT}}{\end{corollaryT}\end{cBox}}	

%	ENVIRONMENT remark
\newenvironment{remark}{\par\vspace{10pt}\small 
% Espacio blanco vertical sobre la nota y tamaño de fuente menor
\begin{list}{}{
\leftmargin=35pt % Indentación sobre la izquierda
\rightmargin=25pt}\item\ignorespaces % Indentación sobre la derecha
\makebox[-2.5pt]{\begin{tikzpicture}[overlay]
\node[draw=ocre!60,line width=1pt,circle,fill=ocre!25,font=\sffamily\bfseries,inner sep=2pt,outer sep=0pt] at (-15pt,0pt){\textcolor{ocre}{N}}; \end{tikzpicture}} % R naranja en un círculo (Pedro)
\advance\baselineskip -1pt}{\end{list}\vskip5pt} 
% Espaciado de línea más estrecho y espacio en blanco después del comentario


\newenvironment{solutionExe}{\par\vspace{10pt}\small 
\begin{list}{}{
\leftmargin=35pt 
\rightmargin=25pt}\item\ignorespaces 
\makebox[-2.5pt]{\begin{tikzpicture}[overlay]
\node[draw=ocre!60,line width=1pt,circle,fill=ocre!25,font=\sffamily\bfseries,inner sep=2pt,outer sep=0pt] at (-15pt,0pt){\textcolor{ocre}{S}}; \end{tikzpicture}} 
\advance\baselineskip -1pt}{\end{list}\vskip5pt} 

\newenvironment{solutionExa}{\par\vspace{10pt}\small 
\begin{list}{}{
\leftmargin=35pt 
\rightmargin=25pt}\item\ignorespaces 
\makebox[-2.5pt]{\begin{tikzpicture}[overlay]
\node[draw=ocre!60,line width=1pt,circle,fill=ocre!55,font=\sffamily\bfseries,inner sep=2pt,outer sep=0pt] at (-15pt,0pt){\textcolor{ocre}{S}}; \end{tikzpicture}} 
\advance\baselineskip -1pt}{\end{list}\vskip5pt} 

\usepackage{tcolorbox}

\usetikzlibrary{trees}

\theoremstyle{ocrenum}
\newtheorem{solutionT}[dummy]{Solución}  % (Pedro: Corollary)
\newenvironment{solution}{\begin{cBox}\begin{solutionT}}{\end{solutionT}\end{cBox}}	


\newcommand{\tcolorboxsolucion}[2]{%
\begin{tcolorbox}[colback=green!5!white,colframe=green!75!black,title=#1] 
 #2
 %\tcblower  % pone una línea discontinua
\end{tcolorbox}
}% final definición comando

\newtcbox{\mybox}[1][green]{on line,
arc=0pt,outer arc=0pt,colback=#1!10!white,colframe=#1!50!black, boxsep=0pt,left=1pt,right=1pt,top=2pt,bottom=2pt, boxrule=0pt,bottomrule=1pt,toprule=1pt}



\mdfdefinestyle{exampledefault}{%
rightline=true,innerleftmargin=10,innerrightmargin=10,
frametitlerule=true,frametitlerulecolor=green,
frametitlebackgroundcolor=yellow,
frametitlerulewidth=2pt}





\newcommand{\betheorem}{\begin{theorem}}
\newcommand{\eetheorem}{\end{theorem}}
\newcommand{\bedefinition}{\begin{definition}}
\newcommand{\eedefinition}{\end{definition}}

\newcommand{\beremark}{\begin{remark}}
\newcommand{\eeremark}{\end{remark}}
\newcommand{\beexercise}{\begin{exercise}}
\newcommand{\eeexercise}{\end{exercise}}
\newcommand{\beexample}{\begin{example}}
\newcommand{\eeexample}{\end{example}}
\newcommand{\becorollary}{\begin{corollary}}
\newcommand{\eecorollary}{\end{corollary}}


\newcommand{\besolutionExe}{\begin{solutionExe}}
\newcommand{\eesolutionExe}{\end{solutionExe}}
\newcommand{\besolutionExa}{\begin{solutionExa}}
\newcommand{\eesolutionExa}{\end{solutionExa}}


%%%%%%%%


% Caja Salida Markdown
\newmdenv[skipabove=7pt,
skipbelow=7pt,
rightline=false,
leftline=true,
topline=false,
bottomline=false,
backgroundcolor=GreenYellow!10,
linecolor=GreenYellow!80,
innerleftmargin=5pt,
innerrightmargin=5pt,
innertopmargin=10pt,%5pt
innerbottommargin=5pt,
leftmargin=0cm,
rightmargin=0cm,
linewidth=4pt]{mBox}	

%% RMarkdown
\newenvironment{markdownsal}{\begin{mBox}}{\end{mBox}}	

\newcommand{\bmarkdownsal}{\begin{markdownsal}}
\newcommand{\emarkdownsal}{\end{markdownsal}}


\usepackage{array}
\usepackage{multirow}
\usepackage{wrapfig}
\usepackage{colortbl}
\usepackage{pdflscape}
\usepackage{tabu}
\usepackage{threeparttable}
\usepackage{subfig} %new
%\usepackage{booktabs,dcolumn,rotating,thumbpdf,longtable}
\usepackage{dcolumn,rotating}  %new
\usepackage[graphicx]{realboxes} %new de: https://stackoverflow.com/questions/51633434/prevent-pagebreak-in-kableextra-landscape-table

%define el interlineado vertical
%\renewcommand{\baselinestretch}{1.5}

%define etiqueta para las Tablas o Cuadros
%\renewcommand\spanishtablename{Tabla}

%%\bibliographystyle{plain} %new no necesario


%%%%%%%%%%%% PARA USO CON biblatex
% \DefineBibliographyStrings{english}{%
%   backrefpage = {ver pag.\adddot},%
%   backrefpages = {ver pags.\adddot}%
% }

% \DefineBibliographyStrings{spanish}{%
%   backrefpage = {ver pag.\adddot},%
%   backrefpages = {ver pags.\adddot}%
% }
% 
% \DeclareFieldFormat{pagerefformat}{\mkbibparens{{\color{red}\mkbibemph{#1}}}}
% \renewbibmacro*{pageref}{%
%   \iflistundef{pageref}
%     {}
%     {\printtext[pagerefformat]{%
%        \ifnumgreater{\value{pageref}}{1}
%          {\bibstring{backrefpages}\ppspace}
%          {\bibstring{backrefpage}\ppspace}%
%        \printlist[pageref][-\value{listtotal}]{pageref}}}}
% 
%%% de kableExtra
\usepackage{booktabs}
\usepackage{longtable}
%\usepackage{array}
%\usepackage{multirow}
%\usepackage{wrapfig}
%\usepackage{float}
%\usepackage{colortbl}
%\usepackage{pdflscape}
%\usepackage{tabu}
%\usepackage{threeparttable}
\usepackage{threeparttablex}
\usepackage[normalem]{ulem}
\usepackage{makecell}
%\usepackage{xcolor}

%%%%%%% fin: latex_preambulo.tex PLLC


\begin{document}

\bibliographystyle{flexbib}



\raggedbottom

\ifdefined\ifprincipal
\else
\setlength{\parindent}{1em}
\pagestyle{fancy}
\setcounter{tocdepth}{4}
\tableofcontents

\fi

\ifdefined\ifdoblecara
\fancyhead{}{}
\fancyhead[LE,RO]{\scriptsize\rightmark}
\fancyfoot[LO,RE]{\scriptsize\slshape \leftmark}
\fancyfoot[C]{}
\fancyfoot[LE,RO]{\footnotesize\thepage}
\else
\fancyhead{}{}
\fancyhead[RO]{\scriptsize\rightmark}
\fancyfoot[LO]{\scriptsize\slshape \leftmark}
\fancyfoot[C]{}
\fancyfoot[RO]{\footnotesize\thepage}
\fi

\renewcommand{\headrulewidth}{0.4pt}
\renewcommand{\footrulewidth}{0.4pt}

\hypertarget{Seccion3}{%
\chapter{Aplicación práctica: el blackjack}\label{Seccion3}}

\hypertarget{Seccion31}{%
\section{Cómo se juega al blacjack y qué tipo de juego
es}\label{Seccion31}}

El BlackJack, también conocido en lenguaje castellano como veintiuno, es
uno de los juegos mas populares en los casinos de todo el mundo junto
con el poker y la ruleta. A lo largo de estos últimos capítulos
buscaremos estudiar en profundidad este juego para encontrar la
estrategia optima con la que jugar para maximizar beneficios (o
minimizar pérdidas). La característica mas importante que convierte la
búsqueda de esta estrategia óptima en algo mas sencillo de lo que
pudiera parecer en un principio, es el hecho de que el crupier (persona
que reparte las cartas y asegura la normalidad del juego en cada mesa)
está obligado a jugar de manera fija y conocida por todos los jugadores.
Solo son los jugadores los que pueden ir tomando decisiones a lo largo
del juego siempre y cuando se sigan las reglas que ahora pasamos a
comentar.

\emph{Reglas del BlackJack}

\begin{itemize}
\item
  Usamos una baraje francesa de 52 cartas, es decir, 4 palos con 13
  cartas cada uno, del 1 al 10 y 3 figuras (las conocidas en España como
  sota, caballo y rey). Las cartas tienen unos valores: las figuras
  valen 10, y el resto de cartas tienen el mismo valor que su número,
  por ejemplo un 6 vale 6, menos el as que puede tomar los valores 1 u
  11 dependiendo de lo que prefiera el jugador (ya comentaremos en que
  situaciones preferirá que valga una cosa u otra.)
\item
  En el juego, a parte del crupier, participan como mucho 7 jugadores.
  Estos jugadores tienen que apostar el dinero antes de recibir la
  primera carta, lo que le da un atractivo al juego distinto al poker,
  en el que una vez recibamos cartas podemos aumentar la apuesta.
\item
  \textbf{Desempeño y estrategia del crupier.} En primer lugar, una vez
  todos los jugadores han hecho su apuestas, el crupier procede a
  repartir las cartas a los jugadores. Una vez que estos han hecho su
  juego, el crupier empieza a darse cartas a si mismo y está obligado a
  plantarse cuando la suma de las cartas que se haya dado sume al menos
  17. Cuando este tenga un as, que ya comentamos que pueden valer 1 u 11
  a gusto, deberá contarlo como 11 si al recibir el as y contarlo con
  valor 11 la suma de sus cartas es al menos 17. Para aclararlo, si
  tiene un 7 y la carta siguiente es un as, este contará como 11 puesto
  que la suma de los dos cartas sumarían 18, mientras que si por ejemplo
  tuviera un 3 y recibiera un as, no lo contará como 11 y su suma será
  4.
\item
  \textbf{Pagos.} Los pagos se basan en el hecho de ``ganar'' o no al
  crupier. Si el crupier se pasa y el jugador no lo hace, recibe una
  cantidad igual a su apuesta. Si no se pasan ninguno, quien mas se
  aproxime a 21 (de ahí el nombre del juego en español), si es el
  jugador recibe otra vez una cantidad igual a lo que ha apostado y si
  es el crupier se queda con el dinero apostado por el jugador. En caso
  de que empaten, no se producen juegos y el jugador recupera su dinero.
\item
  \textbf{BlackJack.} Esta jugada es la más famosa y la que da nombre al
  juego. Es la mano más poderosa y gana a cualquier otra mano que tenga
  el crupier. Si el jugador posee un blackjack y el crupier no, este
  último deberá pagarle al jugador el 150\% de su apuesta.
\item
  Otras posibilidades que tiene el jugador.

  \begin{itemize}
  \tightlist
  \item
    \textbf{Doblarse.} Si la suma de las dos primeras cartas es igual a
    9, 10 u 11, el jugador tiene la posibilidad de doblar su apuesta
    inicial, pero como desventaja solo podrá recibir una carta más.
  \item
    \textbf{Abrirse.} Cuando las dos primeras cartas que recibe el
    jugador tienen el mismo valor, un 10 y una figura o dos 7 por
    ejemplo, este puede separarlas y jugar a dos manos, teniendo que
    apostar en cada una la cantidad inicial que apostó. Si separamos un
    dos ases, al igual que pasaba antes cuando nos podíamos doblar, solo
    recibiremos una carta más en cada mano. Este apartado aunque parece
    ventajoso tiene el inconveniente de no poder abrirnos mas de una vez
    en cada jugada y en el caso de abrirnos con un as, en caso de
    recibir una carta que valga 10 puntos (figura o un 10) esto no
    contará como blackjack y en el caso de que el crupier lo obtuviese
    el jugador perdería el dinero de las dos manos.
  \item
    \textbf{Asegurarse.} En caso de que la primera carta que el crupier
    se reparta a si mismo sea un as, los jugadores tendrán la opción de
    asegurarse y prevenir un posible blackjack del crupier con una
    apuesta adicional de a lo mas el 50\% de la apuesta inicial. Si
    efectivamente ocurre que el crupier obtiene un blackjack, el jugador
    recibe por el seguro el doble de lo que apostó. Este seguro los
    jugadores deberán obtenerlo si así lo desean antes de que el crupier
    reparta la tercera carta al primer jugador que lo desee.
  \end{itemize}
\end{itemize}

\hypertarget{Seccion32}{%
\section{Estudio riguroso. Como jugar según la mano que
tengamos}\label{Seccion32}}

Llamemos \(x\) al valor total de las cartas que posee el jugador.
Consideramos dos tipos de manos que puede poseer un jugador: la que el
total del jugador es único y menor a 21, llamadas ``manos duras'', y la
que el jugador tiene uno o mas ases de manera que el jugador posee dos
cartas con valores menores a 21. A este último tipo de mano la
llamaremos ``manos blandas'' y requieren diferentes estudios a las manos
duras.

Definamos \(D\) como el valor de la carta mas alta que posee el crupier.
Así tomará los valores \(D=2, \cdots, 10, (1,11)\). Sea \(M(D)\) un
entero tal que si la carta mas alta del crupier es \(D\) y el valor
\(x\) es menor que \(M(D)\), el jugador debería pedir una carta más.
Mientras que dicho valor \(x\) sea al menos el valor de \(M(D)\), el
jugador deberá plantarse. De la misma manera definimos \(M^*(D)\) para
el caso de las manos blandas.

La suposición que hacemos es que una buena manera de saber cuando
dejamos de pedir cartas se da cuando tenemos al menos estos números
\(M(D)\) y \(M^*(D)\). Es decir, si hemos dicho que es bueno para
cualquier jugador dejar de pedir cartas si llegamos a ese valor, también
será correcto hacerlo cuando el valor de la mano sea incluso mayor. Esta
suposición suele ser correcta la mayoría de veces salvo en casos
especiales como cuando el hecho de dejar de pedir cartas se da cuando
los jugadores tienen manos bajas con un numero de cartas restantes a
repartir en el mazo bajo.

El primer paso es comparar la esperanza matemática de elegir \(M(D)=x\)
o \(M(D)=x+1\), con \(x\) tomando un valor único sin exceder 21. Para
las manos blandas comparamos \(M^*(D)=x\) con \(M^*(D)=x+1\). En ambos
casos empleamos la misma estrategia con la diferencia de que para el
primer caso una vez llegamos al valor nos paramos y para el otro pedimos
una carta mas. Para el caso de manos duras, comparar las dos esperanzas
es equivalente a comparar \(E_{p,x}\) la esperanza de pararse en un
total de \(x\), con \(E_{d,x}\), la esperanza de un jugador que con un
total de \(x\) pide una carta mas.Para el caso de las manos blandas, en
el primer caso el jugador se para mientras que en el otro pide una o más
cartas. Por ejemplo, en el caso de que tengamos una mano blanda con un
valor de 17 en el primer caso nos plantaríamos, mientras que en el
segundo caso pediríamos una carta mas. Pongamos que es un 5 por lo tanto
nos pasaríamos pero como el valor del as puede ser 1 u 11, seria de
valor 1 en este caso y obtenemos un total de 12, en cuyo caso deberíamos
pedir una carta mas en la mayoría de ocasiones.

A partir de ahora nos centramos en la diferencia de esas dos esperanzas
antes comentadas, \(E_{p,x}-E_{d,x}\) para ver si es mayor o menor que
0. Si fijamos un valor \(x\), la diferencia \(E_{p,x}-E_{d,x}\) es una
función decreciente en x, \(M(D)\) se obtiene como el menor valor de
\(x\) para el que \(E_{p,x}-E_{d,x} < 0\). Esta función es no creciente
siempre salvo casos excepcional como comentamos anteriormente en los que
crece con \(x\).

Definamos ahora \(T\) variable aleatoria como el valor final del
crupier. Sabemos por las reglas que comentamos en la apartado
\(\ref{Seccion31}\) que si ocurre que \(T>21\) o \(T<x\), el jugador
gana (en el caso de que se haya plantado con ese valor \(x\) en sus
cartas), mientras que si \(T=x\) cada uno recupera el dinero apostado, y
si \(x < T \leq 21\) el jugador pierde la apuesta. Así podemos definir
mejor la esperanza antes comentada:

\[
\begin{array}{ccl}
E_{p,x} & = & P(T>21) + P(T<x) - P(x<T \leq 21) \\
        & = & 2P(T>21) -1
\end{array}
\] Para el caso de la esperanza \(E_{d,x}\) definimos una nueva variable
aleatoria \(J\), que es la una que le queda al jugador después de pedir
una carta mas (solo una). En caso de que el total pueda tener dos
valores sin exceder de 21 (caso de tener un as), \(J\) toma el mayor de
los dos valores.

Sabemos por las reglas que la mano del crupier siempre tiene un valor
mayor que 17 por lo que \(T \geq 17\), así que si \(J<17\), solo
ganaríamos en caso de que \(T>21\) y perderíamos para el resto de los
valores de la variable \(T\). Para este caso la esperanza sería:

\[
P(T>21) - P(T \leq 21)= 2P(T>21) -1
\]

Si el valor de las cartas del jugador es \(17 \leq J \leq 21\), la
esperanza quedaría como:

\[
P(T>21) +P(T<J) - P(J < T \leq 21)
\]

Lógicamente si el valor de la mano del jugador es \(J>21\) siempre vamos
a perder y por tanto esa esperanza sería -1, es decir perderíamos cada
euro apostado.

Una pregunta que podríamos hacernos es si estas variables \(T\) y \(J\)
son independientes. Analizando, El valor que tome la variable \(J\)
afecta a la variable \(T\) en solo si descartamos la posibilidad de que
el crupier desvele una carta de valor \(J-x\). Por lo tanto, si asumimos
la independencia de estas dos variables estaríamos cometiendo un pequeño
sesgo en el calculo de la esperanza. De esta manera:

\[
\begin{array}{ccl}
E_{d,x} & = & P(J<17)[2P(T>21)-1] - P(J>21) \\
        &   & + \Sigma_{j=17}^21 P(J=j) [P(T>21) + P(T<j) - P(j<T \leq21)]
\end{array}
\]

Ya tenemos calculadas las dos esperanzas. Restándolas nos queda:

\[
\begin{array}{ccl}
E_{d,x} -E_{p,x} & = & -2P(T<x) - P(T=x) -2P(T>21)P(J>21) \\
                 &   & + 2P(T<J \leq 21)+P(T=J \leq 21)
\end{array}
\]

Si \(T \geq 17\), los primeros dos términos son cero para el caso en que
\(x<17\). Además, \(P(J>21)\) es también cero para el caso de una mano
dura y menos de 12 para valores de una mano blanda. Por lo tanto esa
diferencia de esperanza \(E_{d,x} -E_{p,x} \geq 0\) para manos duras con
\(x<12\) y para manos blandas con \(x<17\), de lo que sacamos que
\(M(D)>11\) y \(M^*(D)>16\), \(\forall D\)

Consideremos ahora esta diferencia de esperanzas para el caso de valores
\(12 \leq x \geq 16\) en el caso de manos duras. Los dos primeros
términos vuelven a ser ceros, mientras que el ultimo lo podemos
reescribir usando la independencia entre \(J\) y \(T\) como usamos
anteriormente.

\[
E_{d,x} -E_{p,x} = -2P(T>21)P(J>21) + \sum_{t=17}^{21} P(T=t)[2P(t<J \leq 21) + P(J=t)]
\]

Ahora introducimos un supuesto, y es que la distribución de probabilidad
de \(J-x\), la única carta que pide el jugador está dada por:

\begin{itemize}
\item
  \(P(J-x=10) = 4/13\), del hecho de que tenemos 3 figuras más el 10 por
  cada palo en la baraja.
\item
  \(P(J-x=i) = 1/13, i=2,\cdots,9, (1,11)\)
\end{itemize}

De esta manera asumimos la suposición de que obtener una carta es
equiprobable. De primera mano podríamos danos cuenta de que esta
suposición es incorrecta en manos individuales, pero es cierta cuando
nos damos cuenta de que tenemos \(52!\) permutaciones posibles de cartas
en la baraja. Así pues, tendríamos lo siguiente:

\(P(J>21) = \frac{1}{13}(x-8)\), para \(x \geq 12\) en manos duras y
\(P(t<J \leq 21)=\frac{1}{13}(21-t)\), \(P(J=t) = \frac{1}{13}\) con
\(t\) tal que \(17 \leq t \leq 21\). Entonces:

\[
E_{d,x} - E_{p,x} = -2/13(x-8)P(T>21) + \sum_{t=17}^{21} 1/13(43-2t)P(T=t)
\]

Para valores de \(x\), \(12 \leq x \leq 16\) no necesitamos hacer esta
diferencia y ahora explicamos el porqué. Si la diferencia anterior la
igualamos a 0, y teniendo en cuenta que como comentamos la función es
decreciente en \(x\), se obtiene una única solución \(x_0\)

\[
x_0 = 8+ \frac{\sum_{t=17}^{21}(21 \frac{1}{2}-t)P(T=t)}{P(T>21)}
\] Así, si \(x_0 < 12\) entonces \(M(D)=12\). Si
\(x_0>16 entonces M(D)>16\) y si \(12 \leq x_0 \leq 16\) entonces
\(M(D) = [x_0]+1\) (parte entera de \(x_0\)). Para un valor dado de
\(P(T>17)\), mas probabilidad tiene el crupier de tener una buena mano,
mas bajo sea el numero en el que el jugador se pare. Por ejemplo, si
\(P(T>21)=2/5\) y \(P(T=18)=3/5\) entonces \(M(D)=14\) mientras que si
\(P(T>21)=2/5\) y \(P(T=19)=3/5\), entonces \(M(D)=12\).

En el caso de que \(x=17\) (en mano dura):

\[
E_{d,17}-E_{p,17} = -18/13P(T>21)- 5/13P(T=17)+ \sum_{t=18}^{21}1/13(43-2t)P(T=t)
\] Evaluando para cada \(t\) la probabilidad \(P(T=t)\) muestra que esa
diferencia en negativa para todo \(D\) y por tanto, \(M(D) \leq 17\).

Para el caso de manos blandas nos queda también el estudio cuando
\(x=17\). En esa situación,

\[
E_{d,17}-E_{p,17} = -1/13P(T=17)+ \sum_{t=18}^{21}1/13(43-2t)P(T=t)
\]

donde evaluando para cada \(t\) otra vez \(P(T=t)\), muestra que la
diferencia es positiva para todo \(D\) y por lo tanto \(M^*(D)>17\).

A partir de ahora, dado que siempre estamos comparando los estudios de
las manos duras con las manos blandas, vamos a continuar el estudio por
separado, analizando bien cada situación para después sintetizar ambos
resultados en uno de carácter general.

\bibliography{bib/library.bib,bib/paquetes.bib}


%


\end{document}
