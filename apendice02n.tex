\documentclass[12pt,a4paper,]{book}
\def\ifdoblecara{} %% set to true
\def\ifprincipal{} %% set to true
\let\ifprincipal\undefined %% set to false
\def\ifcitapandoc{} %% set to true
\let\ifcitapandoc\undefined %% set to false
\usepackage{lmodern}
% sin fontmathfamily
\usepackage{amssymb,amsmath}
\usepackage{ifxetex,ifluatex}
%\usepackage{fixltx2e} % provides \textsubscript %PLLC
\ifnum 0\ifxetex 1\fi\ifluatex 1\fi=0 % if pdftex
  \usepackage[T1]{fontenc}
  \usepackage[utf8]{inputenc}
\else % if luatex or xelatex
  \ifxetex
    \usepackage{mathspec}
  \else
    \usepackage{fontspec}
  \fi
  \defaultfontfeatures{Ligatures=TeX,Scale=MatchLowercase}
\fi
% use upquote if available, for straight quotes in verbatim environments
\IfFileExists{upquote.sty}{\usepackage{upquote}}{}
% use microtype if available
\IfFileExists{microtype.sty}{%
\usepackage{microtype}
\UseMicrotypeSet[protrusion]{basicmath} % disable protrusion for tt fonts
}{}
\usepackage[margin = 2.5cm]{geometry}
\usepackage{hyperref}
\hypersetup{unicode=true,
            pdfauthor={Nombre Completo Autor},
              pdfborder={0 0 0},
              breaklinks=true}
\urlstyle{same}  % don't use monospace font for urls
%
\usepackage[usenames,dvipsnames]{xcolor}  %new PLLC
\usepackage{color}
\usepackage{fancyvrb}
\newcommand{\VerbBar}{|}
\newcommand{\VERB}{\Verb[commandchars=\\\{\}]}
\DefineVerbatimEnvironment{Highlighting}{Verbatim}{commandchars=\\\{\}}
% Add ',fontsize=\small' for more characters per line
\usepackage{framed}
\definecolor{shadecolor}{RGB}{248,248,248}
\newenvironment{Shaded}{\begin{snugshade}}{\end{snugshade}}
\newcommand{\AlertTok}[1]{\textcolor[rgb]{0.94,0.16,0.16}{#1}}
\newcommand{\AnnotationTok}[1]{\textcolor[rgb]{0.56,0.35,0.01}{\textbf{\textit{#1}}}}
\newcommand{\AttributeTok}[1]{\textcolor[rgb]{0.13,0.29,0.53}{#1}}
\newcommand{\BaseNTok}[1]{\textcolor[rgb]{0.00,0.00,0.81}{#1}}
\newcommand{\BuiltInTok}[1]{#1}
\newcommand{\CharTok}[1]{\textcolor[rgb]{0.31,0.60,0.02}{#1}}
\newcommand{\CommentTok}[1]{\textcolor[rgb]{0.56,0.35,0.01}{\textit{#1}}}
\newcommand{\CommentVarTok}[1]{\textcolor[rgb]{0.56,0.35,0.01}{\textbf{\textit{#1}}}}
\newcommand{\ConstantTok}[1]{\textcolor[rgb]{0.56,0.35,0.01}{#1}}
\newcommand{\ControlFlowTok}[1]{\textcolor[rgb]{0.13,0.29,0.53}{\textbf{#1}}}
\newcommand{\DataTypeTok}[1]{\textcolor[rgb]{0.13,0.29,0.53}{#1}}
\newcommand{\DecValTok}[1]{\textcolor[rgb]{0.00,0.00,0.81}{#1}}
\newcommand{\DocumentationTok}[1]{\textcolor[rgb]{0.56,0.35,0.01}{\textbf{\textit{#1}}}}
\newcommand{\ErrorTok}[1]{\textcolor[rgb]{0.64,0.00,0.00}{\textbf{#1}}}
\newcommand{\ExtensionTok}[1]{#1}
\newcommand{\FloatTok}[1]{\textcolor[rgb]{0.00,0.00,0.81}{#1}}
\newcommand{\FunctionTok}[1]{\textcolor[rgb]{0.13,0.29,0.53}{\textbf{#1}}}
\newcommand{\ImportTok}[1]{#1}
\newcommand{\InformationTok}[1]{\textcolor[rgb]{0.56,0.35,0.01}{\textbf{\textit{#1}}}}
\newcommand{\KeywordTok}[1]{\textcolor[rgb]{0.13,0.29,0.53}{\textbf{#1}}}
\newcommand{\NormalTok}[1]{#1}
\newcommand{\OperatorTok}[1]{\textcolor[rgb]{0.81,0.36,0.00}{\textbf{#1}}}
\newcommand{\OtherTok}[1]{\textcolor[rgb]{0.56,0.35,0.01}{#1}}
\newcommand{\PreprocessorTok}[1]{\textcolor[rgb]{0.56,0.35,0.01}{\textit{#1}}}
\newcommand{\RegionMarkerTok}[1]{#1}
\newcommand{\SpecialCharTok}[1]{\textcolor[rgb]{0.81,0.36,0.00}{\textbf{#1}}}
\newcommand{\SpecialStringTok}[1]{\textcolor[rgb]{0.31,0.60,0.02}{#1}}
\newcommand{\StringTok}[1]{\textcolor[rgb]{0.31,0.60,0.02}{#1}}
\newcommand{\VariableTok}[1]{\textcolor[rgb]{0.00,0.00,0.00}{#1}}
\newcommand{\VerbatimStringTok}[1]{\textcolor[rgb]{0.31,0.60,0.02}{#1}}
\newcommand{\WarningTok}[1]{\textcolor[rgb]{0.56,0.35,0.01}{\textbf{\textit{#1}}}}

% PLLC modifica-ini
% PLLC modifica-fin

\IfFileExists{parskip.sty}{%
\usepackage{parskip}
}{% else
\setlength{\parindent}{0pt}
\setlength{\parskip}{6pt plus 2pt minus 1pt}
}
\setlength{\emergencystretch}{3em}  % prevent overfull lines
\providecommand{\tightlist}{%
  \setlength{\itemsep}{0pt}\setlength{\parskip}{0pt}}
\setcounter{secnumdepth}{5}
% Redefines (sub)paragraphs to behave more like sections
\ifx\paragraph\undefined\else
\let\oldparagraph\paragraph
\renewcommand{\paragraph}[1]{\oldparagraph{#1}\mbox{}}
\fi
\ifx\subparagraph\undefined\else
\let\oldsubparagraph\subparagraph
\renewcommand{\subparagraph}[1]{\oldsubparagraph{#1}\mbox{}}
\fi

%%% Use protect on footnotes to avoid problems with footnotes in titles
\let\rmarkdownfootnote\footnote%
\def\footnote{\protect\rmarkdownfootnote}


  \title{}
    \author{Nombre Completo Autor}
      \date{18/11/2021}


%%%%%%% inicio: latex_preambulo.tex PLLC


%% UTILIZA CODIFICACIÓN UTF-8
%% MODIFICARLO CONVENIENTEMENTE PARA USARLO CON OTRAS CODIFICACIONES


%\usepackage[spanish,es-nodecimaldot,es-noshorthands]{babel}
\usepackage[spanish,es-nodecimaldot,es-noshorthands,es-tabla]{babel}
% Ver: es-tabla (en: https://osl.ugr.es/CTAN/macros/latex/contrib/babel-contrib/spanish/spanish.pdf)
% es-tabla (en: https://tex.stackexchange.com/questions/80443/change-the-word-table-in-table-captions)
\usepackage[spanish, plain, datebegin,sortcompress,nocomment,
noabstract]{flexbib}
 
\usepackage{float}
\usepackage{placeins}
\usepackage{fancyhdr}
% Solucion: ! LaTeX Error: Command \counterwithout already defined.
% https://tex.stackexchange.com/questions/425600/latex-error-command-counterwithout-already-defined
\let\counterwithout\relax
\let\counterwithin\relax
\usepackage{chngcntr}
%\usepackage{microtype}  %antes en template PLLC
\usepackage[utf8]{inputenc}
\usepackage[T1]{fontenc} % Usa codificación 8-bit que tiene 256 glyphs

%\usepackage[dvipsnames]{xcolor}
%\usepackage[usenames,dvipsnames]{xcolor}  %new
\usepackage{pdfpages}
%\usepackage{natbib}




% Para portada: latex_paginatitulo_mod_ST02.tex (inicio)
\usepackage{tikz}
\usepackage{epigraph}
\input{portadas/latex_paginatitulo_mod_ST02_add.sty}
% Para portada: latex_paginatitulo_mod_ST02.tex (fin)

% Para portada: latex_paginatitulo_mod_OV01.tex (inicio)
\usepackage{cpimod}
% Para portada: latex_paginatitulo_mod_OV01.tex (fin)

% Para portada: latex_paginatitulo_mod_OV03.tex (inicio)
\usepackage{KTHEEtitlepage}
% Para portada: latex_paginatitulo_mod_OV03.tex (fin)

\renewcommand{\contentsname}{Índice}
\renewcommand{\listfigurename}{Índice de figuras}
\renewcommand{\listtablename}{Índice de tablas}
\newcommand{\bcols}{}
\newcommand{\ecols}{}
\newcommand{\bcol}[1]{\begin{minipage}{#1\linewidth}}
\newcommand{\ecol}{\end{minipage}}
\newcommand{\balertblock}[1]{\begin{alertblock}{#1}}
\newcommand{\ealertblock}{\end{alertblock}}
\newcommand{\bitemize}{\begin{itemize}}
\newcommand{\eitemize}{\end{itemize}}
\newcommand{\benumerate}{\begin{enumerate}}
\newcommand{\eenumerate}{\end{enumerate}}
\newcommand{\saltopagina}{\newpage}
\newcommand{\bcenter}{\begin{center}}
\newcommand{\ecenter}{\end{center}}
\newcommand{\beproof}{\begin{proof}} %new
\newcommand{\eeproof}{\end{proof}} %new
%De: https://texblog.org/2007/11/07/headerfooter-in-latex-with-fancyhdr/
% \fancyhead
% E: Even page
% O: Odd page
% L: Left field
% C: Center field
% R: Right field
% H: Header
% F: Footer
%\fancyhead[CO,CE]{Resultados}

%OPCION 1
% \fancyhead[LE,RO]{\slshape \rightmark}
% \fancyhead[LO,RE]{\slshape \leftmark}
% \fancyfoot[C]{\thepage}
% \renewcommand{\headrulewidth}{0.4pt}
% \renewcommand{\footrulewidth}{0pt}

%OPCION 2
% \fancyhead[LE,RO]{\slshape \rightmark}
% \fancyfoot[LO,RE]{\slshape \leftmark}
% \fancyfoot[LE,RO]{\thepage}
% \renewcommand{\headrulewidth}{0.4pt}
% \renewcommand{\footrulewidth}{0.4pt}
%%%%%%%%%%
\usepackage{calc,amsfonts}
% Elimina la cabecera de páginas impares vacías al finalizar los capítulos
\usepackage{emptypage}
\makeatletter

%\definecolor{ocre}{RGB}{25,25,243} % Define el color azul (naranja) usado para resaltar algunas salidas
\definecolor{ocre}{RGB}{0,0,0} % Define el color a negro (aparece en los teoremas

%\usepackage{calc} 


%era if(csl-refs) con dolares
% metodobib: true


\usepackage{lipsum}

%\usepackage{tikz} % Requerido para dibujar formas personalizadas

%\usepackage{amsmath,amsthm,amssymb,amsfonts}
\usepackage{amsthm}


% Boxed/framed environments
\newtheoremstyle{ocrenumbox}% % Theorem style name
{0pt}% Space above
{0pt}% Space below
{\normalfont}% % Body font
{}% Indent amount
{\small\bf\sffamily\color{ocre}}% % Theorem head font
{\;}% Punctuation after theorem head
{0.25em}% Space after theorem head
{\small\sffamily\color{ocre}\thmname{#1}\nobreakspace\thmnumber{\@ifnotempty{#1}{}\@upn{#2}}% Theorem text (e.g. Theorem 2.1)
\thmnote{\nobreakspace\the\thm@notefont\sffamily\bfseries\color{black}---\nobreakspace#3.}} % Optional theorem note
\renewcommand{\qedsymbol}{$\blacksquare$}% Optional qed square

\newtheoremstyle{blacknumex}% Theorem style name
{5pt}% Space above
{5pt}% Space below
{\normalfont}% Body font
{} % Indent amount
{\small\bf\sffamily}% Theorem head font
{\;}% Punctuation after theorem head
{0.25em}% Space after theorem head
{\small\sffamily{\tiny\ensuremath{\blacksquare}}\nobreakspace\thmname{#1}\nobreakspace\thmnumber{\@ifnotempty{#1}{}\@upn{#2}}% Theorem text (e.g. Theorem 2.1)
\thmnote{\nobreakspace\the\thm@notefont\sffamily\bfseries---\nobreakspace#3.}}% Optional theorem note

\newtheoremstyle{blacknumbox} % Theorem style name
{0pt}% Space above
{0pt}% Space below
{\normalfont}% Body font
{}% Indent amount
{\small\bf\sffamily}% Theorem head font
{\;}% Punctuation after theorem head
{0.25em}% Space after theorem head
{\small\sffamily\thmname{#1}\nobreakspace\thmnumber{\@ifnotempty{#1}{}\@upn{#2}}% Theorem text (e.g. Theorem 2.1)
\thmnote{\nobreakspace\the\thm@notefont\sffamily\bfseries---\nobreakspace#3.}}% Optional theorem note

% Non-boxed/non-framed environments
\newtheoremstyle{ocrenum}% % Theorem style name
{5pt}% Space above
{5pt}% Space below
{\normalfont}% % Body font
{}% Indent amount
{\small\bf\sffamily\color{ocre}}% % Theorem head font
{\;}% Punctuation after theorem head
{0.25em}% Space after theorem head
{\small\sffamily\color{ocre}\thmname{#1}\nobreakspace\thmnumber{\@ifnotempty{#1}{}\@upn{#2}}% Theorem text (e.g. Theorem 2.1)
\thmnote{\nobreakspace\the\thm@notefont\sffamily\bfseries\color{black}---\nobreakspace#3.}} % Optional theorem note
\renewcommand{\qedsymbol}{$\blacksquare$}% Optional qed square
\makeatother



% Define el estilo texto theorem para cada tipo definido anteriormente
\newcounter{dummy} 
\numberwithin{dummy}{section}
\theoremstyle{ocrenumbox}
\newtheorem{theoremeT}[dummy]{Teorema}  % (Pedro: Theorem)
\newtheorem{problem}{Problema}[chapter]  % (Pedro: Problem)
\newtheorem{exerciseT}{Ejercicio}[chapter] % (Pedro: Exercise)
\theoremstyle{blacknumex}
\newtheorem{exampleT}{Ejemplo}[chapter] % (Pedro: Example)
\theoremstyle{blacknumbox}
\newtheorem{vocabulary}{Vocabulario}[chapter]  % (Pedro: Vocabulary)
\newtheorem{definitionT}{Definición}[section]  % (Pedro: Definition)
\newtheorem{corollaryT}[dummy]{Corolario}  % (Pedro: Corollary)
\theoremstyle{ocrenum}
\newtheorem{proposition}[dummy]{Proposición} % (Pedro: Proposition)


\usepackage[framemethod=default]{mdframed}



\newcommand{\intoo}[2]{\mathopen{]}#1\,;#2\mathclose{[}}
\newcommand{\ud}{\mathop{\mathrm{{}d}}\mathopen{}}
\newcommand{\intff}[2]{\mathopen{[}#1\,;#2\mathclose{]}}
\newtheorem{notation}{Notation}[chapter]


\mdfdefinestyle{exampledefault}{%
rightline=true,innerleftmargin=10,innerrightmargin=10,
frametitlerule=true,frametitlerulecolor=green,
frametitlebackgroundcolor=yellow,
frametitlerulewidth=2pt}


% Theorem box
\newmdenv[skipabove=7pt,
skipbelow=7pt,
backgroundcolor=black!5,
linecolor=ocre,
innerleftmargin=5pt,
innerrightmargin=5pt,
innertopmargin=10pt,%5pt
leftmargin=0cm,
rightmargin=0cm,
innerbottommargin=5pt]{tBox}

% Exercise box	  
\newmdenv[skipabove=7pt,
skipbelow=7pt,
rightline=false,
leftline=true,
topline=false,
bottomline=false,
backgroundcolor=ocre!10,
linecolor=ocre,
innerleftmargin=5pt,
innerrightmargin=5pt,
innertopmargin=10pt,%5pt
innerbottommargin=5pt,
leftmargin=0cm,
rightmargin=0cm,
linewidth=4pt]{eBox}	

% Definition box
\newmdenv[skipabove=7pt,
skipbelow=7pt,
rightline=false,
leftline=true,
topline=false,
bottomline=false,
linecolor=ocre,
innerleftmargin=5pt,
innerrightmargin=5pt,
innertopmargin=10pt,%0pt
leftmargin=0cm,
rightmargin=0cm,
linewidth=4pt,
innerbottommargin=0pt]{dBox}	

% Corollary box
\newmdenv[skipabove=7pt,
skipbelow=7pt,
rightline=false,
leftline=true,
topline=false,
bottomline=false,
linecolor=gray,
backgroundcolor=black!5,
innerleftmargin=5pt,
innerrightmargin=5pt,
innertopmargin=10pt,%5pt
leftmargin=0cm,
rightmargin=0cm,
linewidth=4pt,
innerbottommargin=5pt]{cBox}

% Crea un entorno para cada tipo de theorem y le asigna un estilo 
% con ayuda de las cajas coloreadas anteriores
\newenvironment{theorem}{\begin{tBox}\begin{theoremeT}}{\end{theoremeT}\end{tBox}}
\newenvironment{exercise}{\begin{eBox}\begin{exerciseT}}{\hfill{\color{ocre}\tiny\ensuremath{\blacksquare}}\end{exerciseT}\end{eBox}}				  
\newenvironment{definition}{\begin{dBox}\begin{definitionT}}{\end{definitionT}\end{dBox}}	
\newenvironment{example}{\begin{exampleT}}{\hfill{\tiny\ensuremath{\blacksquare}}\end{exampleT}}		
\newenvironment{corollary}{\begin{cBox}\begin{corollaryT}}{\end{corollaryT}\end{cBox}}	

%	ENVIRONMENT remark
\newenvironment{remark}{\par\vspace{10pt}\small 
% Espacio blanco vertical sobre la nota y tamaño de fuente menor
\begin{list}{}{
\leftmargin=35pt % Indentación sobre la izquierda
\rightmargin=25pt}\item\ignorespaces % Indentación sobre la derecha
\makebox[-2.5pt]{\begin{tikzpicture}[overlay]
\node[draw=ocre!60,line width=1pt,circle,fill=ocre!25,font=\sffamily\bfseries,inner sep=2pt,outer sep=0pt] at (-15pt,0pt){\textcolor{ocre}{N}}; \end{tikzpicture}} % R naranja en un círculo (Pedro)
\advance\baselineskip -1pt}{\end{list}\vskip5pt} 
% Espaciado de línea más estrecho y espacio en blanco después del comentario


\newenvironment{solutionExe}{\par\vspace{10pt}\small 
\begin{list}{}{
\leftmargin=35pt 
\rightmargin=25pt}\item\ignorespaces 
\makebox[-2.5pt]{\begin{tikzpicture}[overlay]
\node[draw=ocre!60,line width=1pt,circle,fill=ocre!25,font=\sffamily\bfseries,inner sep=2pt,outer sep=0pt] at (-15pt,0pt){\textcolor{ocre}{S}}; \end{tikzpicture}} 
\advance\baselineskip -1pt}{\end{list}\vskip5pt} 

\newenvironment{solutionExa}{\par\vspace{10pt}\small 
\begin{list}{}{
\leftmargin=35pt 
\rightmargin=25pt}\item\ignorespaces 
\makebox[-2.5pt]{\begin{tikzpicture}[overlay]
\node[draw=ocre!60,line width=1pt,circle,fill=ocre!55,font=\sffamily\bfseries,inner sep=2pt,outer sep=0pt] at (-15pt,0pt){\textcolor{ocre}{S}}; \end{tikzpicture}} 
\advance\baselineskip -1pt}{\end{list}\vskip5pt} 

\usepackage{tcolorbox}

\usetikzlibrary{trees}

\theoremstyle{ocrenum}
\newtheorem{solutionT}[dummy]{Solución}  % (Pedro: Corollary)
\newenvironment{solution}{\begin{cBox}\begin{solutionT}}{\end{solutionT}\end{cBox}}	


\newcommand{\tcolorboxsolucion}[2]{%
\begin{tcolorbox}[colback=green!5!white,colframe=green!75!black,title=#1] 
 #2
 %\tcblower  % pone una línea discontinua
\end{tcolorbox}
}% final definición comando

\newtcbox{\mybox}[1][green]{on line,
arc=0pt,outer arc=0pt,colback=#1!10!white,colframe=#1!50!black, boxsep=0pt,left=1pt,right=1pt,top=2pt,bottom=2pt, boxrule=0pt,bottomrule=1pt,toprule=1pt}



\mdfdefinestyle{exampledefault}{%
rightline=true,innerleftmargin=10,innerrightmargin=10,
frametitlerule=true,frametitlerulecolor=green,
frametitlebackgroundcolor=yellow,
frametitlerulewidth=2pt}





\newcommand{\betheorem}{\begin{theorem}}
\newcommand{\eetheorem}{\end{theorem}}
\newcommand{\bedefinition}{\begin{definition}}
\newcommand{\eedefinition}{\end{definition}}

\newcommand{\beremark}{\begin{remark}}
\newcommand{\eeremark}{\end{remark}}
\newcommand{\beexercise}{\begin{exercise}}
\newcommand{\eeexercise}{\end{exercise}}
\newcommand{\beexample}{\begin{example}}
\newcommand{\eeexample}{\end{example}}
\newcommand{\becorollary}{\begin{corollary}}
\newcommand{\eecorollary}{\end{corollary}}


\newcommand{\besolutionExe}{\begin{solutionExe}}
\newcommand{\eesolutionExe}{\end{solutionExe}}
\newcommand{\besolutionExa}{\begin{solutionExa}}
\newcommand{\eesolutionExa}{\end{solutionExa}}


%%%%%%%%


% Caja Salida Markdown
\newmdenv[skipabove=7pt,
skipbelow=7pt,
rightline=false,
leftline=true,
topline=false,
bottomline=false,
backgroundcolor=GreenYellow!10,
linecolor=GreenYellow!80,
innerleftmargin=5pt,
innerrightmargin=5pt,
innertopmargin=10pt,%5pt
innerbottommargin=5pt,
leftmargin=0cm,
rightmargin=0cm,
linewidth=4pt]{mBox}	

%% RMarkdown
\newenvironment{markdownsal}{\begin{mBox}}{\end{mBox}}	

\newcommand{\bmarkdownsal}{\begin{markdownsal}}
\newcommand{\emarkdownsal}{\end{markdownsal}}


\usepackage{array}
\usepackage{multirow}
\usepackage{wrapfig}
\usepackage{colortbl}
\usepackage{pdflscape}
\usepackage{tabu}
\usepackage{threeparttable}
\usepackage{subfig} %new
%\usepackage{booktabs,dcolumn,rotating,thumbpdf,longtable}
\usepackage{dcolumn,rotating}  %new
\usepackage[graphicx]{realboxes} %new de: https://stackoverflow.com/questions/51633434/prevent-pagebreak-in-kableextra-landscape-table

%define el interlineado vertical
%\renewcommand{\baselinestretch}{1.5}

%define etiqueta para las Tablas o Cuadros
%\renewcommand\spanishtablename{Tabla}

%%\bibliographystyle{plain} %new no necesario


%%%%%%%%%%%% PARA USO CON biblatex
% \DefineBibliographyStrings{english}{%
%   backrefpage = {ver pag.\adddot},%
%   backrefpages = {ver pags.\adddot}%
% }

% \DefineBibliographyStrings{spanish}{%
%   backrefpage = {ver pag.\adddot},%
%   backrefpages = {ver pags.\adddot}%
% }
% 
% \DeclareFieldFormat{pagerefformat}{\mkbibparens{{\color{red}\mkbibemph{#1}}}}
% \renewbibmacro*{pageref}{%
%   \iflistundef{pageref}
%     {}
%     {\printtext[pagerefformat]{%
%        \ifnumgreater{\value{pageref}}{1}
%          {\bibstring{backrefpages}\ppspace}
%          {\bibstring{backrefpage}\ppspace}%
%        \printlist[pageref][-\value{listtotal}]{pageref}}}}
% 
%%% de kableExtra
\usepackage{booktabs}
\usepackage{longtable}
%\usepackage{array}
%\usepackage{multirow}
%\usepackage{wrapfig}
%\usepackage{float}
%\usepackage{colortbl}
%\usepackage{pdflscape}
%\usepackage{tabu}
%\usepackage{threeparttable}
\usepackage{threeparttablex}
\usepackage[normalem]{ulem}
\usepackage{makecell}
%\usepackage{xcolor}

%%%%%%% fin: latex_preambulo.tex PLLC


\begin{document}

\bibliographystyle{flexbib}



\raggedbottom

\ifdefined\ifprincipal
\else
\setlength{\parindent}{1em}
\pagestyle{fancy}
\setcounter{tocdepth}{4}
\tableofcontents

\fi

\ifdefined\ifdoblecara
\fancyhead{}{}
\fancyhead[LE,RO]{\scriptsize\rightmark}
\fancyfoot[LO,RE]{\scriptsize\slshape \leftmark}
\fancyfoot[C]{}
\fancyfoot[LE,RO]{\footnotesize\thepage}
\else
\fancyhead{}{}
\fancyhead[RO]{\scriptsize\rightmark}
\fancyfoot[LO]{\scriptsize\slshape \leftmark}
\fancyfoot[C]{}
\fancyfoot[RO]{\footnotesize\thepage}
\fi

\renewcommand{\headrulewidth}{0.4pt}
\renewcommand{\footrulewidth}{0.4pt}

\hypertarget{apuxe9ndice-codigo-r-utilizado-en-las-simulaciones}{%
\chapter{Apéndice: Codigo R utilizado en las
simulaciones}\label{apuxe9ndice-codigo-r-utilizado-en-las-simulaciones}}

\hypertarget{codigo-para-calcular-la-suma-final-del-crupier}{%
\section{Codigo para calcular la suma final del
crupier}\label{codigo-para-calcular-la-suma-final-del-crupier}}

\begin{Shaded}
\begin{Highlighting}[]
\FunctionTok{library}\NormalTok{(knitr)}
\FunctionTok{library}\NormalTok{(kableExtra)}
\FunctionTok{library}\NormalTok{(tidyverse)}
\NormalTok{Cartas }\OtherTok{\textless{}{-}} \FunctionTok{c}\NormalTok{(}\StringTok{"2"}\NormalTok{,}\StringTok{"3"}\NormalTok{,}\StringTok{"4"}\NormalTok{,}\StringTok{"5"}\NormalTok{,}\StringTok{"6"}\NormalTok{,}\StringTok{"7"}\NormalTok{,}\StringTok{"8"}\NormalTok{,}\StringTok{"9"}\NormalTok{,}\StringTok{"Figura"}\NormalTok{,}\StringTok{"As"}\NormalTok{)}
\NormalTok{Cantidad\_de\_cada\_carta }\OtherTok{\textless{}{-}} \FunctionTok{c}\NormalTok{(}\DecValTok{4}\NormalTok{,}\DecValTok{4}\NormalTok{,}\DecValTok{4}\NormalTok{,}\DecValTok{4}\NormalTok{,}\DecValTok{4}\NormalTok{,}\DecValTok{4}\NormalTok{,}\DecValTok{4}\NormalTok{,}\DecValTok{4}\NormalTok{,}\DecValTok{16}\NormalTok{,}\DecValTok{4}\NormalTok{)}
\NormalTok{Valor\_cartas }\OtherTok{\textless{}{-}} \FunctionTok{c}\NormalTok{(}\DecValTok{2}\NormalTok{,}\DecValTok{3}\NormalTok{,}\DecValTok{4}\NormalTok{,}\DecValTok{5}\NormalTok{,}\DecValTok{6}\NormalTok{,}\DecValTok{7}\NormalTok{,}\DecValTok{8}\NormalTok{,}\DecValTok{9}\NormalTok{,}\DecValTok{10}\NormalTok{,}\DecValTok{11}\NormalTok{)}
\NormalTok{Probabilidades\_sacar\_carta }\OtherTok{=}\NormalTok{ Cantidad\_de\_cada\_carta}\SpecialCharTok{/}\FunctionTok{sum}\NormalTok{(Cantidad\_de\_cada\_carta)}
\NormalTok{Valor\_cartas\_mano\_dura }\OtherTok{=}\FunctionTok{c}\NormalTok{(}\DecValTok{2}\NormalTok{,}\DecValTok{3}\NormalTok{,}\DecValTok{4}\NormalTok{,}\DecValTok{5}\NormalTok{,}\DecValTok{6}\NormalTok{,}\DecValTok{7}\NormalTok{,}\DecValTok{8}\NormalTok{,}\DecValTok{9}\NormalTok{,}\DecValTok{10}\NormalTok{,}\DecValTok{1}\NormalTok{)}
\NormalTok{Valor\_cartas\_mano\_blanda }\OtherTok{=}\NormalTok{ Valor\_cartas}

\NormalTok{encuentra\_as\_en\_mano }\OtherTok{\textless{}{-}} \ControlFlowTok{function}\NormalTok{(mano)\{}
  \ControlFlowTok{if}\NormalTok{ (}\StringTok{"As"} \SpecialCharTok{\%in\%}\NormalTok{ mano) \{}
    \FunctionTok{return}\NormalTok{(}\ConstantTok{TRUE}\NormalTok{)}
\NormalTok{  \} }\ControlFlowTok{else}\NormalTok{ \{}
    \FunctionTok{return}\NormalTok{(}\ConstantTok{FALSE}\NormalTok{)}
\NormalTok{  \}}
\NormalTok{\}}

\NormalTok{cuenta\_ases\_mano }\OtherTok{\textless{}{-}} \ControlFlowTok{function}\NormalTok{(mano)\{}
\NormalTok{  numero\_ases}\OtherTok{=}\DecValTok{0}
  \ControlFlowTok{for}\NormalTok{ (carta }\ControlFlowTok{in}\NormalTok{ mano) \{}
    \ControlFlowTok{if}\NormalTok{ (carta}\SpecialCharTok{==}\StringTok{"As"}\NormalTok{) \{}
\NormalTok{      numero\_ases}\OtherTok{=}\NormalTok{numero\_ases}\SpecialCharTok{+}\DecValTok{1}
\NormalTok{    \}}
\NormalTok{  \}}
  \FunctionTok{return}\NormalTok{(numero\_ases)}
\NormalTok{\}}


\NormalTok{Devuelve\_salida }\OtherTok{\textless{}{-}} \ControlFlowTok{function}\NormalTok{(suma,mano)\{}
  \ControlFlowTok{if}\NormalTok{ ((}\FunctionTok{all}\NormalTok{(}\FunctionTok{sort}\NormalTok{(mano) }\SpecialCharTok{==} \FunctionTok{sort}\NormalTok{(}\FunctionTok{c}\NormalTok{(}\StringTok{"Figura"}\NormalTok{, }\StringTok{"As"}\NormalTok{))))) \{}
    \FunctionTok{return}\NormalTok{(}\StringTok{"BlackJack"}\NormalTok{)}
\NormalTok{  \} }\ControlFlowTok{else}\NormalTok{\{}
    \ControlFlowTok{if}\NormalTok{ (suma }\SpecialCharTok{\textgreater{}}\DecValTok{21}\NormalTok{) \{}
      \FunctionTok{return}\NormalTok{(}\StringTok{"Se pasa"}\NormalTok{)}
\NormalTok{    \} }\ControlFlowTok{else}\NormalTok{\{}
      \FunctionTok{return}\NormalTok{(}\FunctionTok{as.character}\NormalTok{(suma))}
\NormalTok{    \}}
\NormalTok{  \} }
\NormalTok{\} }


\NormalTok{Calcular\_suma\_mano\_crupier }\OtherTok{\textless{}{-}} \ControlFlowTok{function}\NormalTok{(mano)\{}
\NormalTok{  suma}\OtherTok{=}\DecValTok{0}
  \ControlFlowTok{if}\NormalTok{ ((}\FunctionTok{all}\NormalTok{(}\FunctionTok{sort}\NormalTok{(mano) }\SpecialCharTok{==} \FunctionTok{sort}\NormalTok{(}\FunctionTok{c}\NormalTok{(}\StringTok{"Figura"}\NormalTok{, }\StringTok{"As"}\NormalTok{))))) \{}
\NormalTok{    suma }\OtherTok{=} \DecValTok{21}
\NormalTok{  \} }\ControlFlowTok{else}\NormalTok{ \{}
    \ControlFlowTok{if}\NormalTok{ (}\FunctionTok{encuentra\_as\_en\_mano}\NormalTok{(mano)) \{}
\NormalTok{      mano\_sin\_ases }\OtherTok{\textless{}{-}}\NormalTok{ mano[mano }\SpecialCharTok{!=} \StringTok{"As"}\NormalTok{]}
\NormalTok{      mano\_solo\_ases }\OtherTok{\textless{}{-}}\NormalTok{ mano[mano }\SpecialCharTok{==} \StringTok{"As"}\NormalTok{]}
      \ControlFlowTok{for}\NormalTok{ (carta }\ControlFlowTok{in}\NormalTok{ mano\_sin\_ases) \{}
\NormalTok{        suma}\OtherTok{=}\NormalTok{ suma }\SpecialCharTok{+}\NormalTok{ Valor\_cartas[}\FunctionTok{which}\NormalTok{(Cartas }\SpecialCharTok{==}\NormalTok{ carta)]}
\NormalTok{      \}}
      \ControlFlowTok{while}\NormalTok{ (}\FunctionTok{cuenta\_ases\_mano}\NormalTok{(mano\_solo\_ases)}\SpecialCharTok{\textgreater{}}\DecValTok{0}\NormalTok{) \{}
        \ControlFlowTok{if}\NormalTok{ ((suma }\SpecialCharTok{+} \DecValTok{11}\SpecialCharTok{\textgreater{}=}\DecValTok{17}\NormalTok{) }\SpecialCharTok{\&\&}\NormalTok{ (suma }\SpecialCharTok{+} \DecValTok{11} \SpecialCharTok{\textless{}=} \DecValTok{21}\NormalTok{)) \{}
\NormalTok{          suma }\OtherTok{=}\NormalTok{ suma}\SpecialCharTok{+}\DecValTok{11}
\NormalTok{        \} }\ControlFlowTok{else}\NormalTok{\{}
\NormalTok{          suma}\OtherTok{=}\NormalTok{suma}\SpecialCharTok{+}\DecValTok{1}\NormalTok{\}}
\NormalTok{        mano\_solo\_ases }\OtherTok{=}\NormalTok{mano\_solo\_ases[}\SpecialCharTok{{-}}\DecValTok{1}\NormalTok{]}
\NormalTok{      \}}
\NormalTok{    \} }
    \ControlFlowTok{else}\NormalTok{ \{}
      \ControlFlowTok{for}\NormalTok{ (carta }\ControlFlowTok{in}\NormalTok{ mano) \{}
\NormalTok{        suma}\OtherTok{=}\NormalTok{ suma }\SpecialCharTok{+}\NormalTok{ Valor\_cartas[}\FunctionTok{which}\NormalTok{(Cartas }\SpecialCharTok{==}\NormalTok{ carta)]}
\NormalTok{      \}}
\NormalTok{    \}}
\NormalTok{  \}}
  \FunctionTok{return}\NormalTok{(suma)}
\NormalTok{\}}

\NormalTok{Mano\_dada\_crupier }\OtherTok{\textless{}{-}} \ControlFlowTok{function}\NormalTok{(b)\{ }\CommentTok{\#b es la carta que todos ven que tiene el crupier}
\NormalTok{  suma}\OtherTok{=}\DecValTok{0}
\NormalTok{  mano}\OtherTok{=}\FunctionTok{c}\NormalTok{(b)}
  \ControlFlowTok{while}\NormalTok{ (suma}\SpecialCharTok{\textless{}}\DecValTok{17}\NormalTok{) \{}
\NormalTok{    carta\_1 }\OtherTok{\textless{}{-}} \FunctionTok{sample}\NormalTok{(Cartas,}\AttributeTok{size =} \DecValTok{1}\NormalTok{,}\AttributeTok{prob =}\NormalTok{ Probabilidades\_sacar\_carta)  }
\NormalTok{    mano }\OtherTok{=} \FunctionTok{c}\NormalTok{(mano,carta\_1)}
\NormalTok{    suma}\OtherTok{=}\FunctionTok{Calcular\_suma\_mano\_crupier}\NormalTok{(mano)}
\NormalTok{  \}}
  \FunctionTok{Devuelve\_salida}\NormalTok{(suma,mano)}
\NormalTok{\}}

\NormalTok{resultados\_tabla }\OtherTok{\textless{}{-}} \FunctionTok{data.frame}\NormalTok{()}

\CommentTok{\# Calcular probabilidades para cada carta visible}
\ControlFlowTok{for}\NormalTok{ (carta }\ControlFlowTok{in}\NormalTok{ Cartas) \{}
\NormalTok{  salida }\OtherTok{\textless{}{-}} \FunctionTok{replicate}\NormalTok{(}\DecValTok{10000}\NormalTok{, }\FunctionTok{Mano\_dada\_crupier}\NormalTok{(carta))}
\NormalTok{  tabla }\OtherTok{\textless{}{-}} \FunctionTok{prop.table}\NormalTok{(}\FunctionTok{table}\NormalTok{(}\FunctionTok{factor}\NormalTok{(salida, }
                                   \AttributeTok{levels =} \FunctionTok{c}\NormalTok{(}\StringTok{"17"}\NormalTok{,}\StringTok{"18"}\NormalTok{,}\StringTok{"19"}\NormalTok{,}\StringTok{"20"}\NormalTok{,}\StringTok{"21"}\NormalTok{,}\StringTok{"BlackJack"}\NormalTok{,}\StringTok{"Se pasa"}\NormalTok{))))}
\NormalTok{  resultados\_tabla }\OtherTok{\textless{}{-}} \FunctionTok{rbind}\NormalTok{(resultados\_tabla, }\FunctionTok{as.numeric}\NormalTok{(tabla))}
  \FunctionTok{colnames}\NormalTok{(resultados\_tabla)}\OtherTok{=}\FunctionTok{c}\NormalTok{(}\StringTok{"17"}\NormalTok{,}\StringTok{"18"}\NormalTok{,}\StringTok{"19"}\NormalTok{,}\StringTok{"20"}\NormalTok{,}\StringTok{"21"}\NormalTok{,}\StringTok{"BlackJack"}\NormalTok{,}\StringTok{"Se pasa"}\NormalTok{)}
\NormalTok{\}}
\FunctionTok{rownames}\NormalTok{(resultados\_tabla)}\OtherTok{=}\FunctionTok{c}\NormalTok{(}\StringTok{"2"}\NormalTok{,}\StringTok{"3"}\NormalTok{,}\StringTok{"4"}\NormalTok{,}\StringTok{"5"}\NormalTok{,}\StringTok{"6"}\NormalTok{,}\StringTok{"7"}\NormalTok{,}\StringTok{"8"}\NormalTok{,}\StringTok{"9"}\NormalTok{,}\StringTok{"Figura"}\NormalTok{,}\StringTok{"As"}\NormalTok{)}


\NormalTok{Calculo\_G\_0 }\OtherTok{\textless{}{-}} \ControlFlowTok{function}\NormalTok{(x, b, tabla, }\AttributeTok{es\_blackjack =} \ConstantTok{FALSE}\NormalTok{) \{}
\NormalTok{  probabilidades\_final\_crupier }\OtherTok{\textless{}{-}}\NormalTok{ tabla[b, ]}
  
  \CommentTok{\# Extraemos probabilidades}
\NormalTok{  pr\_T\_17\_20 }\OtherTok{\textless{}{-}} \FunctionTok{as.numeric}\NormalTok{(probabilidades\_final\_crupier[}\FunctionTok{c}\NormalTok{(}\StringTok{"17"}\NormalTok{, }\StringTok{"18"}\NormalTok{, }\StringTok{"19"}\NormalTok{, }\StringTok{"20"}\NormalTok{)])}
\NormalTok{  pr\_T\_21 }\OtherTok{\textless{}{-}}\NormalTok{ probabilidades\_final\_crupier}\SpecialCharTok{$}\StringTok{"21"}
\NormalTok{  pr\_T\_bj }\OtherTok{\textless{}{-}}\NormalTok{ probabilidades\_final\_crupier}\SpecialCharTok{$}\StringTok{"BlackJack"}
\NormalTok{  pr\_T\_pasa }\OtherTok{\textless{}{-}}\NormalTok{ probabilidades\_final\_crupier}\SpecialCharTok{$}\StringTok{\textasciigrave{}}\AttributeTok{Se pasa}\StringTok{\textasciigrave{}}
  
  \ControlFlowTok{if}\NormalTok{ (x }\SpecialCharTok{\textgreater{}} \DecValTok{21}\NormalTok{) \{}
    \FunctionTok{return}\NormalTok{(}\SpecialCharTok{{-}}\DecValTok{1}\NormalTok{)}
\NormalTok{  \}}
  
  \ControlFlowTok{if}\NormalTok{ (x }\SpecialCharTok{==} \DecValTok{21} \SpecialCharTok{\&\&}\NormalTok{ es\_blackjack) \{}
    \FunctionTok{return}\NormalTok{(}\FloatTok{1.5} \SpecialCharTok{*}\NormalTok{ (}\DecValTok{1} \SpecialCharTok{{-}}\NormalTok{ pr\_T\_bj))  }\CommentTok{\# 0 * pr\_T\_bj es innecesario}
\NormalTok{  \}}
  
\NormalTok{  pr\_T\_menor }\OtherTok{\textless{}{-}} \FunctionTok{sum}\NormalTok{(pr\_T\_17\_20[}\FunctionTok{which}\NormalTok{(}\DecValTok{17}\SpecialCharTok{:}\DecValTok{20} \SpecialCharTok{\textless{}}\NormalTok{ x)])}
\NormalTok{  pr\_T\_igual }\OtherTok{\textless{}{-}} \FunctionTok{sum}\NormalTok{(pr\_T\_17\_20[}\FunctionTok{which}\NormalTok{(}\DecValTok{17}\SpecialCharTok{:}\DecValTok{20} \SpecialCharTok{==}\NormalTok{ x)]) }\SpecialCharTok{+} \FunctionTok{ifelse}\NormalTok{(x }\SpecialCharTok{==} \DecValTok{21}\NormalTok{, pr\_T\_21, }\DecValTok{0}\NormalTok{)}
\NormalTok{  pr\_T\_mayor }\OtherTok{\textless{}{-}} \FunctionTok{sum}\NormalTok{(pr\_T\_17\_20[}\FunctionTok{which}\NormalTok{(}\DecValTok{17}\SpecialCharTok{:}\DecValTok{20} \SpecialCharTok{\textgreater{}}\NormalTok{ x)]) }\SpecialCharTok{+} \FunctionTok{ifelse}\NormalTok{(x }\SpecialCharTok{\textless{}} \DecValTok{21}\NormalTok{, pr\_T\_21, }\DecValTok{0}\NormalTok{) }\SpecialCharTok{+}\NormalTok{ pr\_T\_bj}
  
\NormalTok{  ganancia }\OtherTok{\textless{}{-}}\NormalTok{ (}\SpecialCharTok{+}\DecValTok{1}\NormalTok{) }\SpecialCharTok{*}\NormalTok{ (pr\_T\_menor }\SpecialCharTok{+}\NormalTok{ pr\_T\_pasa) }\SpecialCharTok{+}\NormalTok{ (}\DecValTok{0}\NormalTok{) }\SpecialCharTok{*}\NormalTok{ pr\_T\_igual }\SpecialCharTok{+}\NormalTok{ (}\SpecialCharTok{{-}}\DecValTok{1}\NormalTok{) }\SpecialCharTok{*}\NormalTok{ pr\_T\_mayor}
  
  \FunctionTok{return}\NormalTok{(ganancia)}
\NormalTok{\}}
\end{Highlighting}
\end{Shaded}

\hypertarget{codigo-para-la-ganancia-y-estrategia-en-caso-de-mano-dura}{%
\section{Codigo para la ganancia y estrategia en caso de mano
dura}\label{codigo-para-la-ganancia-y-estrategia-en-caso-de-mano-dura}}

\begin{Shaded}
\begin{Highlighting}[]
\NormalTok{estrategia\_G\_optima }\OtherTok{\textless{}{-}} \FunctionTok{matrix}\NormalTok{(}\StringTok{""}\NormalTok{, }\AttributeTok{nrow =} \DecValTok{28}\NormalTok{, }\AttributeTok{ncol =} \DecValTok{10}\NormalTok{)}
\FunctionTok{rownames}\NormalTok{(estrategia\_G\_optima) }\OtherTok{\textless{}{-}} \FunctionTok{as.character}\NormalTok{(}\DecValTok{4}\SpecialCharTok{:}\DecValTok{31}\NormalTok{)}
\FunctionTok{colnames}\NormalTok{(estrategia\_G\_optima) }\OtherTok{\textless{}{-}}\NormalTok{ Cartas  }\CommentTok{\# del 2 al As}
\NormalTok{ganancia\_G\_optima }\OtherTok{\textless{}{-}} \FunctionTok{matrix}\NormalTok{(}\ConstantTok{NA}\NormalTok{, }\AttributeTok{nrow =} \DecValTok{28}\NormalTok{, }\AttributeTok{ncol =} \DecValTok{10}\NormalTok{)}
\FunctionTok{rownames}\NormalTok{(ganancia\_G\_optima) }\OtherTok{\textless{}{-}} \FunctionTok{as.character}\NormalTok{(}\DecValTok{4}\SpecialCharTok{:}\DecValTok{31}\NormalTok{)}
\FunctionTok{colnames}\NormalTok{(ganancia\_G\_optima) }\OtherTok{\textless{}{-}}\NormalTok{ Cartas  }\CommentTok{\# del 2 al As}
\ControlFlowTok{for}\NormalTok{ (x }\ControlFlowTok{in} \DecValTok{22}\SpecialCharTok{:}\DecValTok{31}\NormalTok{) \{}
\NormalTok{  ganancia\_G\_optima[}\FunctionTok{as.character}\NormalTok{(x),] }\OtherTok{\textless{}{-}} \SpecialCharTok{{-}}\DecValTok{1}
\NormalTok{  estrategia\_G\_optima[}\FunctionTok{as.character}\NormalTok{(x),] }\OtherTok{\textless{}{-}} \StringTok{"Parar"}
\NormalTok{\}}

\ControlFlowTok{for}\NormalTok{ (x }\ControlFlowTok{in} \DecValTok{21}\SpecialCharTok{:}\DecValTok{4}\NormalTok{) \{}
  \ControlFlowTok{for}\NormalTok{ (b }\ControlFlowTok{in}\NormalTok{ Cartas) \{}
    
    \CommentTok{\# Verifica si es blackjack (21 con 2 cartas), solo posible si x == 21}
\NormalTok{    es\_blackjack }\OtherTok{\textless{}{-}}\NormalTok{ (x }\SpecialCharTok{==} \DecValTok{21}\NormalTok{)  }\CommentTok{\# Aquí podrías añadir verificación con número de cartas}
    
\NormalTok{    G0 }\OtherTok{\textless{}{-}} \FunctionTok{Calculo\_G\_0}\NormalTok{(x, b, resultados\_tabla,}\AttributeTok{es\_blackjack =}\NormalTok{ es\_blackjack)}
    
    \CommentTok{\# Esperanza de continuar}
\NormalTok{    G\_continuar }\OtherTok{\textless{}{-}} \DecValTok{0}
    \ControlFlowTok{for}\NormalTok{ (j }\ControlFlowTok{in} \DecValTok{1}\SpecialCharTok{:}\DecValTok{10}\NormalTok{) \{}
\NormalTok{      nueva\_x }\OtherTok{\textless{}{-}}\NormalTok{ x }\SpecialCharTok{+}\NormalTok{ Valor\_cartas\_mano\_dura[j]}
      \ControlFlowTok{if}\NormalTok{ (nueva\_x }\SpecialCharTok{\textgreater{}} \DecValTok{21}\NormalTok{) \{}
\NormalTok{        G\_continuar }\OtherTok{\textless{}{-}}\NormalTok{ G\_continuar }\SpecialCharTok{{-}}\NormalTok{ Probabilidades\_sacar\_carta[j]}
\NormalTok{      \} }\ControlFlowTok{else}\NormalTok{ \{}
\NormalTok{        G\_continuar }\OtherTok{\textless{}{-}}\NormalTok{ G\_continuar }\SpecialCharTok{+}\NormalTok{ Probabilidades\_sacar\_carta[j] }\SpecialCharTok{*}\NormalTok{ganancia\_G\_optima[}\FunctionTok{as.character}\NormalTok{(nueva\_x),b] }
\NormalTok{      \}}
\NormalTok{    \}}
    
    \ControlFlowTok{if}\NormalTok{ (G0 }\SpecialCharTok{\textgreater{}=}\NormalTok{ G\_continuar) \{}
\NormalTok{      estrategia\_G\_optima[}\FunctionTok{as.character}\NormalTok{(x), b] }\OtherTok{\textless{}{-}} \StringTok{"Parar"}
\NormalTok{      ganancia\_G\_optima[}\FunctionTok{as.character}\NormalTok{(x),b] }\OtherTok{\textless{}{-}}\NormalTok{ G0}
\NormalTok{    \} }\ControlFlowTok{else}\NormalTok{ \{}
\NormalTok{      estrategia\_G\_optima[}\FunctionTok{as.character}\NormalTok{(x), b] }\OtherTok{\textless{}{-}} \StringTok{"Continuar"}
\NormalTok{      ganancia\_G\_optima[}\FunctionTok{as.character}\NormalTok{(x),b] }\OtherTok{\textless{}{-}}\NormalTok{ G\_continuar}
\NormalTok{    \}}
\NormalTok{  \}}
\NormalTok{\}}

\FunctionTok{kable}\NormalTok{(estrategia\_G\_optima, }
      \AttributeTok{caption =} \StringTok{"Tabla de procedimientos si el jugador posee una mano dura"}\NormalTok{,}
      \AttributeTok{align =} \StringTok{"c"}\NormalTok{) }\SpecialCharTok{\%\textgreater{}\%}
  \FunctionTok{kable\_styling}\NormalTok{(}\AttributeTok{bootstrap\_options =} \FunctionTok{c}\NormalTok{(}\StringTok{"striped"}\NormalTok{, }\StringTok{"hover"}\NormalTok{, }\StringTok{"condensed"}\NormalTok{),}
                \AttributeTok{full\_width =}\NormalTok{ F, }\AttributeTok{font\_size =} \DecValTok{12}\NormalTok{)}

\NormalTok{knitr}\SpecialCharTok{::}\FunctionTok{kable}\NormalTok{(}\FunctionTok{round}\NormalTok{(ganancia\_G\_optima,}\DecValTok{3}\NormalTok{), }
      \AttributeTok{caption =} \StringTok{"Tabla de ganancias si el jugador posee una mano dura"}\NormalTok{,}
      \AttributeTok{align =} \StringTok{"c"}\NormalTok{) }\SpecialCharTok{\%\textgreater{}\%}
  \FunctionTok{kable\_styling}\NormalTok{(}\AttributeTok{bootstrap\_options =} \FunctionTok{c}\NormalTok{(}\StringTok{"striped"}\NormalTok{, }\StringTok{"hover"}\NormalTok{, }\StringTok{"condensed"}\NormalTok{),}
                \AttributeTok{full\_width =}\NormalTok{ F, }\AttributeTok{font\_size =} \DecValTok{12}\NormalTok{)}
\end{Highlighting}
\end{Shaded}

\hypertarget{codigo-para-la-ganancia-y-estrategia-en-caso-de-mano-blanda}{%
\section{Codigo para la ganancia y estrategia en caso de mano
blanda}\label{codigo-para-la-ganancia-y-estrategia-en-caso-de-mano-blanda}}

\begin{Shaded}
\begin{Highlighting}[]
\NormalTok{estrategia\_G\_optima\_mano\_blanda }\OtherTok{\textless{}{-}} \FunctionTok{matrix}\NormalTok{(}\StringTok{""}\NormalTok{, }\AttributeTok{nrow =} \DecValTok{28}\NormalTok{, }\AttributeTok{ncol =} \DecValTok{10}\NormalTok{)}
\FunctionTok{rownames}\NormalTok{(estrategia\_G\_optima\_mano\_blanda) }\OtherTok{\textless{}{-}} \FunctionTok{as.character}\NormalTok{(}\DecValTok{4}\SpecialCharTok{:}\DecValTok{31}\NormalTok{)}
\FunctionTok{colnames}\NormalTok{(estrategia\_G\_optima\_mano\_blanda) }\OtherTok{\textless{}{-}}\NormalTok{ Cartas  }\CommentTok{\# del 2 al As}
\NormalTok{ganancia\_G\_optima\_mano\_blanda }\OtherTok{\textless{}{-}} \FunctionTok{matrix}\NormalTok{(}\ConstantTok{NA}\NormalTok{, }\AttributeTok{nrow =} \DecValTok{28}\NormalTok{, }\AttributeTok{ncol =} \DecValTok{10}\NormalTok{)}
\FunctionTok{rownames}\NormalTok{(ganancia\_G\_optima\_mano\_blanda) }\OtherTok{\textless{}{-}} \FunctionTok{as.character}\NormalTok{(}\DecValTok{4}\SpecialCharTok{:}\DecValTok{31}\NormalTok{)}
\FunctionTok{colnames}\NormalTok{(ganancia\_G\_optima\_mano\_blanda) }\OtherTok{\textless{}{-}}\NormalTok{ Cartas  }\CommentTok{\# del 2 al As}
\ControlFlowTok{for}\NormalTok{ (x }\ControlFlowTok{in} \DecValTok{22}\SpecialCharTok{:}\DecValTok{31}\NormalTok{) \{}
\NormalTok{  ganancia\_G\_optima\_mano\_blanda[}\FunctionTok{as.character}\NormalTok{(x),] }\OtherTok{\textless{}{-}} \SpecialCharTok{{-}}\DecValTok{1}
\NormalTok{  estrategia\_G\_optima\_mano\_blanda[}\FunctionTok{as.character}\NormalTok{(x),] }\OtherTok{\textless{}{-}} \StringTok{"Parar"}
\NormalTok{\}}

\ControlFlowTok{for}\NormalTok{ (x }\ControlFlowTok{in} \DecValTok{21}\SpecialCharTok{:}\DecValTok{4}\NormalTok{) \{}
  \ControlFlowTok{for}\NormalTok{ (b }\ControlFlowTok{in}\NormalTok{ Cartas) \{}
    
    \CommentTok{\# Verifica si es blackjack (21 con 2 cartas), solo posible si x == 21}
\NormalTok{    es\_blackjack }\OtherTok{\textless{}{-}}\NormalTok{ (x }\SpecialCharTok{==} \DecValTok{21}\NormalTok{)  }\CommentTok{\# Aquí podrías añadir verificación con número de cartas}
    
\NormalTok{    G0 }\OtherTok{\textless{}{-}} \FunctionTok{Calculo\_G\_0}\NormalTok{(x, b, resultados\_tabla,}\AttributeTok{es\_blackjack =}\NormalTok{ es\_blackjack)}
    
    \CommentTok{\# Esperanza de continuar}
\NormalTok{    G\_continuar }\OtherTok{\textless{}{-}} \DecValTok{0}
    \ControlFlowTok{for}\NormalTok{ (j }\ControlFlowTok{in} \DecValTok{1}\SpecialCharTok{:}\DecValTok{10}\NormalTok{) \{}
\NormalTok{      nueva\_x }\OtherTok{\textless{}{-}}\NormalTok{ x }\SpecialCharTok{+}\NormalTok{ Valor\_cartas\_mano\_blanda[j]}
      \ControlFlowTok{if}\NormalTok{ (nueva\_x }\SpecialCharTok{\textgreater{}} \DecValTok{21}\NormalTok{) \{}
\NormalTok{        nueva\_x\_dura }\OtherTok{\textless{}{-}}\NormalTok{ nueva\_x}\DecValTok{{-}10}
\NormalTok{        G\_continuar }\OtherTok{\textless{}{-}}\NormalTok{ G\_continuar }\SpecialCharTok{{-}}\NormalTok{ Probabilidades\_sacar\_carta[j]}\SpecialCharTok{*}\NormalTok{ganancia\_G\_optima[}\FunctionTok{as.character}\NormalTok{(nueva\_x\_dura),b]}
\NormalTok{      \} }\ControlFlowTok{else}\NormalTok{ \{}
\NormalTok{        G\_continuar }\OtherTok{\textless{}{-}}\NormalTok{ G\_continuar }\SpecialCharTok{+}\NormalTok{ Probabilidades\_sacar\_carta[j] }\SpecialCharTok{*}\NormalTok{ganancia\_G\_optima\_mano\_blanda[}\FunctionTok{as.character}\NormalTok{(nueva\_x),b] }
\NormalTok{      \}}
\NormalTok{    \}}
    
    \ControlFlowTok{if}\NormalTok{ (G0 }\SpecialCharTok{\textgreater{}=}\NormalTok{ G\_continuar) \{}
\NormalTok{      estrategia\_G\_optima\_mano\_blanda[}\FunctionTok{as.character}\NormalTok{(x), b] }\OtherTok{\textless{}{-}} \StringTok{"Parar"}
\NormalTok{      ganancia\_G\_optima\_mano\_blanda[}\FunctionTok{as.character}\NormalTok{(x),b] }\OtherTok{\textless{}{-}}\NormalTok{ G0}
\NormalTok{    \} }\ControlFlowTok{else}\NormalTok{ \{}
\NormalTok{      estrategia\_G\_optima\_mano\_blanda[}\FunctionTok{as.character}\NormalTok{(x), b] }\OtherTok{\textless{}{-}} \StringTok{"Continuar"}
\NormalTok{      ganancia\_G\_optima\_mano\_blanda[}\FunctionTok{as.character}\NormalTok{(x),b] }\OtherTok{\textless{}{-}}\NormalTok{ G\_continuar}
\NormalTok{    \}}
\NormalTok{  \}}
\NormalTok{\}}

\FunctionTok{kable}\NormalTok{(estrategia\_G\_optima\_mano\_blanda, }
      \AttributeTok{caption =} \StringTok{"Tabla de procedimientos si el jugador posee una mano blanda"}\NormalTok{,}
      \AttributeTok{align =} \StringTok{"c"}\NormalTok{) }\SpecialCharTok{\%\textgreater{}\%}
  \FunctionTok{kable\_styling}\NormalTok{(}\AttributeTok{bootstrap\_options =} \FunctionTok{c}\NormalTok{(}\StringTok{"striped"}\NormalTok{, }\StringTok{"hover"}\NormalTok{, }\StringTok{"condensed"}\NormalTok{),}
                \AttributeTok{full\_width =}\NormalTok{ F, }\AttributeTok{font\_size =} \DecValTok{12}\NormalTok{)}

\FunctionTok{kable}\NormalTok{(}\FunctionTok{round}\NormalTok{(ganancia\_G\_optima\_mano\_blanda,}\DecValTok{5}\NormalTok{), }
      \AttributeTok{caption =} \StringTok{"Tabla de ganancias si el jugador posee una mano blanda"}\NormalTok{,}
      \AttributeTok{align =} \StringTok{"c"}\NormalTok{) }\SpecialCharTok{\%\textgreater{}\%}
  \FunctionTok{kable\_styling}\NormalTok{(}\AttributeTok{bootstrap\_options =} \FunctionTok{c}\NormalTok{(}\StringTok{"striped"}\NormalTok{, }\StringTok{"hover"}\NormalTok{, }\StringTok{"condensed"}\NormalTok{),}
                \AttributeTok{full\_width =}\NormalTok{ F, }\AttributeTok{font\_size =} \DecValTok{12}\NormalTok{)}
\end{Highlighting}
\end{Shaded}

\hypertarget{codigo-para-la-ganancia-y-estrategia-en-caso-de-doblarse-o-no}{%
\section{Codigo para la ganancia y estrategia en caso de doblarse o
no}\label{codigo-para-la-ganancia-y-estrategia-en-caso-de-doblarse-o-no}}

\begin{Shaded}
\begin{Highlighting}[]
\NormalTok{estrategia\_doblarse\_NoDoblarse }\OtherTok{\textless{}{-}} \FunctionTok{matrix}\NormalTok{(}\StringTok{""}\NormalTok{, }\AttributeTok{nrow =} \DecValTok{3}\NormalTok{, }\AttributeTok{ncol =} \DecValTok{10}\NormalTok{)}
\FunctionTok{rownames}\NormalTok{(estrategia\_doblarse\_NoDoblarse) }\OtherTok{\textless{}{-}} \FunctionTok{as.character}\NormalTok{(}\DecValTok{9}\SpecialCharTok{:}\DecValTok{11}\NormalTok{)}
\FunctionTok{colnames}\NormalTok{(estrategia\_doblarse\_NoDoblarse) }\OtherTok{\textless{}{-}}\NormalTok{ Cartas  }\CommentTok{\# del 2 al As}
\NormalTok{ganancia\_doblarse\_NoDoblarse }\OtherTok{\textless{}{-}} \FunctionTok{matrix}\NormalTok{(}\ConstantTok{NA}\NormalTok{, }\AttributeTok{nrow =} \DecValTok{3}\NormalTok{, }\AttributeTok{ncol =} \DecValTok{10}\NormalTok{)}
\FunctionTok{rownames}\NormalTok{(ganancia\_doblarse\_NoDoblarse) }\OtherTok{\textless{}{-}} \FunctionTok{as.character}\NormalTok{(}\DecValTok{9}\SpecialCharTok{:}\DecValTok{11}\NormalTok{)}
\FunctionTok{colnames}\NormalTok{(ganancia\_doblarse\_NoDoblarse) }\OtherTok{\textless{}{-}}\NormalTok{ Cartas  }\CommentTok{\# del 2 al As}

\ControlFlowTok{for}\NormalTok{ (x }\ControlFlowTok{in} \DecValTok{9}\SpecialCharTok{:}\DecValTok{11}\NormalTok{) \{}
  \ControlFlowTok{for}\NormalTok{ (b }\ControlFlowTok{in}\NormalTok{ Cartas) \{}
\NormalTok{    G\_estrella}\OtherTok{=}\NormalTok{ganancia\_G\_optima[}\FunctionTok{as.character}\NormalTok{(x),b]}
    \CommentTok{\# Verifica si es blackjack (21 con 2 cartas), solo posible si x == 21}
\NormalTok{    es\_blackjack }\OtherTok{\textless{}{-}}\NormalTok{ (x }\SpecialCharTok{==} \DecValTok{21}\NormalTok{)  }\CommentTok{\# Aquí podrías añadir verificación con número de cartas}
    \CommentTok{\# Esperanza de continuar}
\NormalTok{    G\_continuar }\OtherTok{\textless{}{-}} \DecValTok{0}
    \ControlFlowTok{for}\NormalTok{ (j }\ControlFlowTok{in} \DecValTok{1}\SpecialCharTok{:}\DecValTok{10}\NormalTok{) \{}
\NormalTok{      nueva\_x }\OtherTok{\textless{}{-}}\NormalTok{ x }\SpecialCharTok{+}\NormalTok{ Valor\_cartas\_mano\_blanda[j]}
\NormalTok{      G\_continuar }\OtherTok{\textless{}{-}}\NormalTok{ G\_continuar }\SpecialCharTok{+}\NormalTok{ Probabilidades\_sacar\_carta[j] }\SpecialCharTok{*} \FunctionTok{Calculo\_G\_0}\NormalTok{(nueva\_x,b,resultados\_tabla,}\AttributeTok{es\_blackjack =}\NormalTok{ es\_blackjack) }
\NormalTok{    \}}
    \ControlFlowTok{if}\NormalTok{ (}\DecValTok{2}\SpecialCharTok{*}\NormalTok{G\_continuar }\SpecialCharTok{\textgreater{}}\NormalTok{ G\_estrella) \{}
\NormalTok{      estrategia\_doblarse\_NoDoblarse[}\FunctionTok{as.character}\NormalTok{(x), b] }\OtherTok{\textless{}{-}} \StringTok{"Doblarse"}
\NormalTok{      ganancia\_doblarse\_NoDoblarse [}\FunctionTok{as.character}\NormalTok{(x),b] }\OtherTok{\textless{}{-}} \DecValTok{2}\SpecialCharTok{*}\NormalTok{G\_continuar}
\NormalTok{    \} }\ControlFlowTok{else}\NormalTok{ \{}
\NormalTok{      estrategia\_doblarse\_NoDoblarse[}\FunctionTok{as.character}\NormalTok{(x), b] }\OtherTok{\textless{}{-}} \StringTok{"No doblarse"}
\NormalTok{      ganancia\_doblarse\_NoDoblarse [}\FunctionTok{as.character}\NormalTok{(x),b] }\OtherTok{\textless{}{-}}\NormalTok{ G\_estrella}
\NormalTok{    \}}
\NormalTok{  \}}
\NormalTok{\}}

\FunctionTok{kable}\NormalTok{(estrategia\_doblarse\_NoDoblarse, }
      \AttributeTok{caption =} \StringTok{"Tabla de procedimientos para decidir si doblarse o no"}\NormalTok{,}
      \AttributeTok{align =} \StringTok{"c"}\NormalTok{) }\SpecialCharTok{\%\textgreater{}\%}
  \FunctionTok{kable\_styling}\NormalTok{(}\AttributeTok{bootstrap\_options =} \FunctionTok{c}\NormalTok{(}\StringTok{"striped"}\NormalTok{, }\StringTok{"hover"}\NormalTok{, }\StringTok{"condensed"}\NormalTok{),}
                \AttributeTok{full\_width =}\NormalTok{ F, }\AttributeTok{font\_size =} \DecValTok{12}\NormalTok{)}

\FunctionTok{kable}\NormalTok{(}\FunctionTok{round}\NormalTok{(ganancia\_doblarse\_NoDoblarse,}\DecValTok{5}\NormalTok{), }
      \AttributeTok{caption =} \StringTok{"Tabla de ganancias al doblarse o no hacerlo"}\NormalTok{,}
      \AttributeTok{align =} \StringTok{"c"}\NormalTok{) }\SpecialCharTok{\%\textgreater{}\%}
  \FunctionTok{kable\_styling}\NormalTok{(}\AttributeTok{bootstrap\_options =} \FunctionTok{c}\NormalTok{(}\StringTok{"striped"}\NormalTok{, }\StringTok{"hover"}\NormalTok{, }\StringTok{"condensed"}\NormalTok{),}
                \AttributeTok{full\_width =}\NormalTok{ F, }\AttributeTok{font\_size =} \DecValTok{12}\NormalTok{)}
\end{Highlighting}
\end{Shaded}

\hypertarget{codigo-para-la-ganancia-y-estrategia-en-caso-de-doblarse-o-no-1}{%
\section{Codigo para la ganancia y estrategia en caso de doblarse o
no}\label{codigo-para-la-ganancia-y-estrategia-en-caso-de-doblarse-o-no-1}}

\begin{Shaded}
\begin{Highlighting}[]
\NormalTok{estrategia\_abrirse\_NoAbrirse }\OtherTok{\textless{}{-}} \FunctionTok{matrix}\NormalTok{(}\StringTok{""}\NormalTok{, }\AttributeTok{nrow =} \DecValTok{10}\NormalTok{, }\AttributeTok{ncol =} \DecValTok{10}\NormalTok{)}
\FunctionTok{rownames}\NormalTok{(estrategia\_abrirse\_NoAbrirse) }\OtherTok{\textless{}{-}} \FunctionTok{as.character}\NormalTok{(}\DecValTok{2}\SpecialCharTok{:}\DecValTok{11}\NormalTok{)}
\FunctionTok{colnames}\NormalTok{(estrategia\_abrirse\_NoAbrirse) }\OtherTok{\textless{}{-}}\NormalTok{ Cartas  }\CommentTok{\# del 2 al As}
\NormalTok{ganancia\_abrirse\_NoAbrirse }\OtherTok{\textless{}{-}} \FunctionTok{matrix}\NormalTok{(}\ConstantTok{NA}\NormalTok{, }\AttributeTok{nrow =} \DecValTok{10}\NormalTok{, }\AttributeTok{ncol =} \DecValTok{10}\NormalTok{)}
\FunctionTok{rownames}\NormalTok{(ganancia\_abrirse\_NoAbrirse) }\OtherTok{\textless{}{-}} \FunctionTok{as.character}\NormalTok{(}\DecValTok{2}\SpecialCharTok{:}\DecValTok{11}\NormalTok{)}
\FunctionTok{colnames}\NormalTok{(ganancia\_abrirse\_NoAbrirse) }\OtherTok{\textless{}{-}}\NormalTok{ Cartas  }\CommentTok{\# del 2 al As}

\ControlFlowTok{for}\NormalTok{ (z }\ControlFlowTok{in} \DecValTok{2}\SpecialCharTok{:}\DecValTok{10}\NormalTok{) \{}
  \ControlFlowTok{for}\NormalTok{ (b }\ControlFlowTok{in}\NormalTok{ Cartas) \{}
\NormalTok{    G\_estrella}\OtherTok{=}\NormalTok{ganancia\_G\_optima[}\FunctionTok{as.character}\NormalTok{(}\DecValTok{2}\SpecialCharTok{*}\NormalTok{z),b]}
    \CommentTok{\# Esperanza de continuar}
\NormalTok{    G\_continuar }\OtherTok{\textless{}{-}} \DecValTok{0}
    \ControlFlowTok{for}\NormalTok{ (j }\ControlFlowTok{in} \DecValTok{1}\SpecialCharTok{:}\DecValTok{10}\NormalTok{) \{}
\NormalTok{      nueva\_z }\OtherTok{\textless{}{-}}\NormalTok{ z }\SpecialCharTok{+}\NormalTok{ Valor\_cartas\_mano\_blanda[j]}
\NormalTok{      G\_continuar }\OtherTok{\textless{}{-}}\NormalTok{ G\_continuar }\SpecialCharTok{+}\NormalTok{ Probabilidades\_sacar\_carta[j] }\SpecialCharTok{*}\NormalTok{ ganancia\_G\_optima[}\FunctionTok{as.character}\NormalTok{(nueva\_z),b] }
\NormalTok{    \}}
    \ControlFlowTok{if}\NormalTok{ (}\DecValTok{2}\SpecialCharTok{*}\NormalTok{G\_continuar }\SpecialCharTok{\textgreater{}}\NormalTok{ G\_estrella) \{}
\NormalTok{      estrategia\_abrirse\_NoAbrirse[}\FunctionTok{as.character}\NormalTok{(z), b] }\OtherTok{\textless{}{-}} \StringTok{"Abrirse"}
\NormalTok{      ganancia\_abrirse\_NoAbrirse[}\FunctionTok{as.character}\NormalTok{(z),b] }\OtherTok{\textless{}{-}} \DecValTok{2}\SpecialCharTok{*}\NormalTok{G\_continuar}
\NormalTok{    \} }\ControlFlowTok{else}\NormalTok{ \{}
\NormalTok{      estrategia\_abrirse\_NoAbrirse[}\FunctionTok{as.character}\NormalTok{(z), b] }\OtherTok{\textless{}{-}} \StringTok{"No abrirse"}
\NormalTok{      ganancia\_abrirse\_NoAbrirse[}\FunctionTok{as.character}\NormalTok{(z),b] }\OtherTok{\textless{}{-}}\NormalTok{ G\_estrella}
\NormalTok{    \}}
\NormalTok{  \}}
\NormalTok{\}}

\ControlFlowTok{for}\NormalTok{ (b }\ControlFlowTok{in}\NormalTok{ Cartas) \{}
\NormalTok{  G\_estrella}\OtherTok{=}\NormalTok{ganancia\_G\_optima\_mano\_blanda[}\FunctionTok{as.character}\NormalTok{(}\DecValTok{12}\NormalTok{),b]}
  \CommentTok{\# Esperanza de continuar}
\NormalTok{  G\_continuar }\OtherTok{\textless{}{-}} \DecValTok{0}
  \ControlFlowTok{for}\NormalTok{ (j }\ControlFlowTok{in} \DecValTok{1}\SpecialCharTok{:}\DecValTok{10}\NormalTok{) \{}
\NormalTok{    nueva\_z }\OtherTok{\textless{}{-}}\NormalTok{ z }\SpecialCharTok{+}\NormalTok{ Valor\_cartas\_mano\_blanda[j]}
\NormalTok{    G\_continuar }\OtherTok{\textless{}{-}}\NormalTok{ G\_continuar }\SpecialCharTok{+}\NormalTok{ Probabilidades\_sacar\_carta[j] }\SpecialCharTok{*} \FunctionTok{Calculo\_G\_0}\NormalTok{(nueva\_z,b,}\AttributeTok{tabla =}\NormalTok{ resultados\_tabla)}
\NormalTok{  \}}
  \ControlFlowTok{if}\NormalTok{ (}\DecValTok{2}\SpecialCharTok{*}\NormalTok{G\_continuar }\SpecialCharTok{\textgreater{}}\NormalTok{ G\_estrella) \{}
\NormalTok{    estrategia\_abrirse\_NoAbrirse[}\FunctionTok{as.character}\NormalTok{(}\DecValTok{11}\NormalTok{), b] }\OtherTok{\textless{}{-}} \StringTok{"Abrirse"}
\NormalTok{    ganancia\_abrirse\_NoAbrirse[}\FunctionTok{as.character}\NormalTok{(}\DecValTok{11}\NormalTok{),b] }\OtherTok{\textless{}{-}} \DecValTok{2}\SpecialCharTok{*}\NormalTok{G\_continuar}
\NormalTok{  \} }\ControlFlowTok{else}\NormalTok{ \{}
\NormalTok{    estrategia\_abrirse\_NoAbrirse[}\FunctionTok{as.character}\NormalTok{(}\DecValTok{11}\NormalTok{), b] }\OtherTok{\textless{}{-}} \StringTok{"No abrirse"}
\NormalTok{    ganancia\_abrirse\_NoAbrirse[}\FunctionTok{as.character}\NormalTok{(}\DecValTok{11}\NormalTok{),b] }\OtherTok{\textless{}{-}}\NormalTok{ G\_estrella}
\NormalTok{  \}}
\NormalTok{\}}
\FunctionTok{rownames}\NormalTok{(estrategia\_abrirse\_NoAbrirse) }\OtherTok{\textless{}{-}} \FunctionTok{paste0}\NormalTok{(Cartas,}\StringTok{"{-}"}\NormalTok{,Cartas)}
\FunctionTok{rownames}\NormalTok{(ganancia\_abrirse\_NoAbrirse) }\OtherTok{\textless{}{-}} \FunctionTok{paste0}\NormalTok{(Cartas,}\StringTok{"{-}"}\NormalTok{,Cartas)}

\FunctionTok{kable}\NormalTok{(estrategia\_abrirse\_NoAbrirse, }
      \AttributeTok{caption =} \StringTok{"Tabla de procedimientos para decidir si abrirse o no"}\NormalTok{,}
      \AttributeTok{align =} \StringTok{"c"}\NormalTok{) }\SpecialCharTok{\%\textgreater{}\%}
\FunctionTok{kable\_styling}\NormalTok{(}\AttributeTok{bootstrap\_options =} \FunctionTok{c}\NormalTok{(}\StringTok{"striped"}\NormalTok{, }\StringTok{"hover"}\NormalTok{, }\StringTok{"condensed"}\NormalTok{),}
              \AttributeTok{full\_width =}\NormalTok{ F, }\AttributeTok{font\_size =} \DecValTok{12}\NormalTok{)}

\FunctionTok{kable}\NormalTok{(}\FunctionTok{round}\NormalTok{(ganancia\_abrirse\_NoAbrirse,}\DecValTok{5}\NormalTok{), }
      \AttributeTok{caption =} \StringTok{"Tabla de ganancias al abrirse o no hacerlo"}\NormalTok{,}
      \AttributeTok{align =} \StringTok{"c"}\NormalTok{) }\SpecialCharTok{\%\textgreater{}\%}
\FunctionTok{kable\_styling}\NormalTok{(}\AttributeTok{bootstrap\_options =} \FunctionTok{c}\NormalTok{(}\StringTok{"striped"}\NormalTok{, }\StringTok{"hover"}\NormalTok{, }\StringTok{"condensed"}\NormalTok{),}
              \AttributeTok{full\_width =}\NormalTok{ F, }\AttributeTok{font\_size =} \DecValTok{12}\NormalTok{)}
\end{Highlighting}
\end{Shaded}


\bibliography{bib/library.bib,bib/paquetes.bib}


%


\end{document}
