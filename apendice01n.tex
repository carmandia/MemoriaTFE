\documentclass[12pt,a4paper,]{book}
\def\ifdoblecara{} %% set to true
\def\ifprincipal{} %% set to true
\let\ifprincipal\undefined %% set to false
\def\ifcitapandoc{} %% set to true
\let\ifcitapandoc\undefined %% set to false
\usepackage{lmodern}
% sin fontmathfamily
\usepackage{amssymb,amsmath}
\usepackage{ifxetex,ifluatex}
%\usepackage{fixltx2e} % provides \textsubscript %PLLC
\ifnum 0\ifxetex 1\fi\ifluatex 1\fi=0 % if pdftex
  \usepackage[T1]{fontenc}
  \usepackage[utf8]{inputenc}
\else % if luatex or xelatex
  \ifxetex
    \usepackage{mathspec}
  \else
    \usepackage{fontspec}
  \fi
  \defaultfontfeatures{Ligatures=TeX,Scale=MatchLowercase}
\fi
% use upquote if available, for straight quotes in verbatim environments
\IfFileExists{upquote.sty}{\usepackage{upquote}}{}
% use microtype if available
\IfFileExists{microtype.sty}{%
\usepackage{microtype}
\UseMicrotypeSet[protrusion]{basicmath} % disable protrusion for tt fonts
}{}
\usepackage[margin = 2.5cm]{geometry}
\usepackage{hyperref}
\hypersetup{unicode=true,
            pdfauthor={Nombre Completo Autor},
              pdfborder={0 0 0},
              breaklinks=true}
\urlstyle{same}  % don't use monospace font for urls
%
\usepackage[usenames,dvipsnames]{xcolor}  %new PLLC
\usepackage{color}
\usepackage{fancyvrb}
\newcommand{\VerbBar}{|}
\newcommand{\VERB}{\Verb[commandchars=\\\{\}]}
\DefineVerbatimEnvironment{Highlighting}{Verbatim}{commandchars=\\\{\}}
% Add ',fontsize=\small' for more characters per line
\usepackage{framed}
\definecolor{shadecolor}{RGB}{248,248,248}
\newenvironment{Shaded}{\begin{snugshade}}{\end{snugshade}}
\newcommand{\AlertTok}[1]{\textcolor[rgb]{0.94,0.16,0.16}{#1}}
\newcommand{\AnnotationTok}[1]{\textcolor[rgb]{0.56,0.35,0.01}{\textbf{\textit{#1}}}}
\newcommand{\AttributeTok}[1]{\textcolor[rgb]{0.13,0.29,0.53}{#1}}
\newcommand{\BaseNTok}[1]{\textcolor[rgb]{0.00,0.00,0.81}{#1}}
\newcommand{\BuiltInTok}[1]{#1}
\newcommand{\CharTok}[1]{\textcolor[rgb]{0.31,0.60,0.02}{#1}}
\newcommand{\CommentTok}[1]{\textcolor[rgb]{0.56,0.35,0.01}{\textit{#1}}}
\newcommand{\CommentVarTok}[1]{\textcolor[rgb]{0.56,0.35,0.01}{\textbf{\textit{#1}}}}
\newcommand{\ConstantTok}[1]{\textcolor[rgb]{0.56,0.35,0.01}{#1}}
\newcommand{\ControlFlowTok}[1]{\textcolor[rgb]{0.13,0.29,0.53}{\textbf{#1}}}
\newcommand{\DataTypeTok}[1]{\textcolor[rgb]{0.13,0.29,0.53}{#1}}
\newcommand{\DecValTok}[1]{\textcolor[rgb]{0.00,0.00,0.81}{#1}}
\newcommand{\DocumentationTok}[1]{\textcolor[rgb]{0.56,0.35,0.01}{\textbf{\textit{#1}}}}
\newcommand{\ErrorTok}[1]{\textcolor[rgb]{0.64,0.00,0.00}{\textbf{#1}}}
\newcommand{\ExtensionTok}[1]{#1}
\newcommand{\FloatTok}[1]{\textcolor[rgb]{0.00,0.00,0.81}{#1}}
\newcommand{\FunctionTok}[1]{\textcolor[rgb]{0.13,0.29,0.53}{\textbf{#1}}}
\newcommand{\ImportTok}[1]{#1}
\newcommand{\InformationTok}[1]{\textcolor[rgb]{0.56,0.35,0.01}{\textbf{\textit{#1}}}}
\newcommand{\KeywordTok}[1]{\textcolor[rgb]{0.13,0.29,0.53}{\textbf{#1}}}
\newcommand{\NormalTok}[1]{#1}
\newcommand{\OperatorTok}[1]{\textcolor[rgb]{0.81,0.36,0.00}{\textbf{#1}}}
\newcommand{\OtherTok}[1]{\textcolor[rgb]{0.56,0.35,0.01}{#1}}
\newcommand{\PreprocessorTok}[1]{\textcolor[rgb]{0.56,0.35,0.01}{\textit{#1}}}
\newcommand{\RegionMarkerTok}[1]{#1}
\newcommand{\SpecialCharTok}[1]{\textcolor[rgb]{0.81,0.36,0.00}{\textbf{#1}}}
\newcommand{\SpecialStringTok}[1]{\textcolor[rgb]{0.31,0.60,0.02}{#1}}
\newcommand{\StringTok}[1]{\textcolor[rgb]{0.31,0.60,0.02}{#1}}
\newcommand{\VariableTok}[1]{\textcolor[rgb]{0.00,0.00,0.00}{#1}}
\newcommand{\VerbatimStringTok}[1]{\textcolor[rgb]{0.31,0.60,0.02}{#1}}
\newcommand{\WarningTok}[1]{\textcolor[rgb]{0.56,0.35,0.01}{\textbf{\textit{#1}}}}

% PLLC modifica-ini
% PLLC modifica-fin

\IfFileExists{parskip.sty}{%
\usepackage{parskip}
}{% else
\setlength{\parindent}{0pt}
\setlength{\parskip}{6pt plus 2pt minus 1pt}
}
\setlength{\emergencystretch}{3em}  % prevent overfull lines
\providecommand{\tightlist}{%
  \setlength{\itemsep}{0pt}\setlength{\parskip}{0pt}}
\setcounter{secnumdepth}{5}
% Redefines (sub)paragraphs to behave more like sections
\ifx\paragraph\undefined\else
\let\oldparagraph\paragraph
\renewcommand{\paragraph}[1]{\oldparagraph{#1}\mbox{}}
\fi
\ifx\subparagraph\undefined\else
\let\oldsubparagraph\subparagraph
\renewcommand{\subparagraph}[1]{\oldsubparagraph{#1}\mbox{}}
\fi

%%% Use protect on footnotes to avoid problems with footnotes in titles
\let\rmarkdownfootnote\footnote%
\def\footnote{\protect\rmarkdownfootnote}


  \title{}
    \author{Nombre Completo Autor}
      \date{18/11/2021}


%%%%%%% inicio: latex_preambulo.tex PLLC


%% UTILIZA CODIFICACIÓN UTF-8
%% MODIFICARLO CONVENIENTEMENTE PARA USARLO CON OTRAS CODIFICACIONES


%\usepackage[spanish,es-nodecimaldot,es-noshorthands]{babel}
\usepackage[spanish,es-nodecimaldot,es-noshorthands,es-tabla]{babel}
% Ver: es-tabla (en: https://osl.ugr.es/CTAN/macros/latex/contrib/babel-contrib/spanish/spanish.pdf)
% es-tabla (en: https://tex.stackexchange.com/questions/80443/change-the-word-table-in-table-captions)
\usepackage[spanish, plain, datebegin,sortcompress,nocomment,
noabstract]{flexbib}
 
\usepackage{float}
\usepackage{placeins}
\usepackage{fancyhdr}
% Solucion: ! LaTeX Error: Command \counterwithout already defined.
% https://tex.stackexchange.com/questions/425600/latex-error-command-counterwithout-already-defined
\let\counterwithout\relax
\let\counterwithin\relax
\usepackage{chngcntr}
%\usepackage{microtype}  %antes en template PLLC
\usepackage[utf8]{inputenc}
\usepackage[T1]{fontenc} % Usa codificación 8-bit que tiene 256 glyphs

%\usepackage[dvipsnames]{xcolor}
%\usepackage[usenames,dvipsnames]{xcolor}  %new
\usepackage{pdfpages}
%\usepackage{natbib}




% Para portada: latex_paginatitulo_mod_ST02.tex (inicio)
\usepackage{tikz}
\usepackage{epigraph}
\input{portadas/latex_paginatitulo_mod_ST02_add.sty}
% Para portada: latex_paginatitulo_mod_ST02.tex (fin)

% Para portada: latex_paginatitulo_mod_OV01.tex (inicio)
\usepackage{cpimod}
% Para portada: latex_paginatitulo_mod_OV01.tex (fin)

% Para portada: latex_paginatitulo_mod_OV03.tex (inicio)
\usepackage{KTHEEtitlepage}
% Para portada: latex_paginatitulo_mod_OV03.tex (fin)

\renewcommand{\contentsname}{Índice}
\renewcommand{\listfigurename}{Índice de figuras}
\renewcommand{\listtablename}{Índice de tablas}
\newcommand{\bcols}{}
\newcommand{\ecols}{}
\newcommand{\bcol}[1]{\begin{minipage}{#1\linewidth}}
\newcommand{\ecol}{\end{minipage}}
\newcommand{\balertblock}[1]{\begin{alertblock}{#1}}
\newcommand{\ealertblock}{\end{alertblock}}
\newcommand{\bitemize}{\begin{itemize}}
\newcommand{\eitemize}{\end{itemize}}
\newcommand{\benumerate}{\begin{enumerate}}
\newcommand{\eenumerate}{\end{enumerate}}
\newcommand{\saltopagina}{\newpage}
\newcommand{\bcenter}{\begin{center}}
\newcommand{\ecenter}{\end{center}}
\newcommand{\beproof}{\begin{proof}} %new
\newcommand{\eeproof}{\end{proof}} %new
%De: https://texblog.org/2007/11/07/headerfooter-in-latex-with-fancyhdr/
% \fancyhead
% E: Even page
% O: Odd page
% L: Left field
% C: Center field
% R: Right field
% H: Header
% F: Footer
%\fancyhead[CO,CE]{Resultados}

%OPCION 1
% \fancyhead[LE,RO]{\slshape \rightmark}
% \fancyhead[LO,RE]{\slshape \leftmark}
% \fancyfoot[C]{\thepage}
% \renewcommand{\headrulewidth}{0.4pt}
% \renewcommand{\footrulewidth}{0pt}

%OPCION 2
% \fancyhead[LE,RO]{\slshape \rightmark}
% \fancyfoot[LO,RE]{\slshape \leftmark}
% \fancyfoot[LE,RO]{\thepage}
% \renewcommand{\headrulewidth}{0.4pt}
% \renewcommand{\footrulewidth}{0.4pt}
%%%%%%%%%%
\usepackage{calc,amsfonts}
% Elimina la cabecera de páginas impares vacías al finalizar los capítulos
\usepackage{emptypage}
\makeatletter

%\definecolor{ocre}{RGB}{25,25,243} % Define el color azul (naranja) usado para resaltar algunas salidas
\definecolor{ocre}{RGB}{0,0,0} % Define el color a negro (aparece en los teoremas

%\usepackage{calc} 


%era if(csl-refs) con dolares
% metodobib: true


\usepackage{lipsum}

%\usepackage{tikz} % Requerido para dibujar formas personalizadas

%\usepackage{amsmath,amsthm,amssymb,amsfonts}
\usepackage{amsthm}


% Boxed/framed environments
\newtheoremstyle{ocrenumbox}% % Theorem style name
{0pt}% Space above
{0pt}% Space below
{\normalfont}% % Body font
{}% Indent amount
{\small\bf\sffamily\color{ocre}}% % Theorem head font
{\;}% Punctuation after theorem head
{0.25em}% Space after theorem head
{\small\sffamily\color{ocre}\thmname{#1}\nobreakspace\thmnumber{\@ifnotempty{#1}{}\@upn{#2}}% Theorem text (e.g. Theorem 2.1)
\thmnote{\nobreakspace\the\thm@notefont\sffamily\bfseries\color{black}---\nobreakspace#3.}} % Optional theorem note
\renewcommand{\qedsymbol}{$\blacksquare$}% Optional qed square

\newtheoremstyle{blacknumex}% Theorem style name
{5pt}% Space above
{5pt}% Space below
{\normalfont}% Body font
{} % Indent amount
{\small\bf\sffamily}% Theorem head font
{\;}% Punctuation after theorem head
{0.25em}% Space after theorem head
{\small\sffamily{\tiny\ensuremath{\blacksquare}}\nobreakspace\thmname{#1}\nobreakspace\thmnumber{\@ifnotempty{#1}{}\@upn{#2}}% Theorem text (e.g. Theorem 2.1)
\thmnote{\nobreakspace\the\thm@notefont\sffamily\bfseries---\nobreakspace#3.}}% Optional theorem note

\newtheoremstyle{blacknumbox} % Theorem style name
{0pt}% Space above
{0pt}% Space below
{\normalfont}% Body font
{}% Indent amount
{\small\bf\sffamily}% Theorem head font
{\;}% Punctuation after theorem head
{0.25em}% Space after theorem head
{\small\sffamily\thmname{#1}\nobreakspace\thmnumber{\@ifnotempty{#1}{}\@upn{#2}}% Theorem text (e.g. Theorem 2.1)
\thmnote{\nobreakspace\the\thm@notefont\sffamily\bfseries---\nobreakspace#3.}}% Optional theorem note

% Non-boxed/non-framed environments
\newtheoremstyle{ocrenum}% % Theorem style name
{5pt}% Space above
{5pt}% Space below
{\normalfont}% % Body font
{}% Indent amount
{\small\bf\sffamily\color{ocre}}% % Theorem head font
{\;}% Punctuation after theorem head
{0.25em}% Space after theorem head
{\small\sffamily\color{ocre}\thmname{#1}\nobreakspace\thmnumber{\@ifnotempty{#1}{}\@upn{#2}}% Theorem text (e.g. Theorem 2.1)
\thmnote{\nobreakspace\the\thm@notefont\sffamily\bfseries\color{black}---\nobreakspace#3.}} % Optional theorem note
\renewcommand{\qedsymbol}{$\blacksquare$}% Optional qed square
\makeatother



% Define el estilo texto theorem para cada tipo definido anteriormente
\newcounter{dummy} 
\numberwithin{dummy}{section}
\theoremstyle{ocrenumbox}
\newtheorem{theoremeT}[dummy]{Teorema}  % (Pedro: Theorem)
\newtheorem{problem}{Problema}[chapter]  % (Pedro: Problem)
\newtheorem{exerciseT}{Ejercicio}[chapter] % (Pedro: Exercise)
\theoremstyle{blacknumex}
\newtheorem{exampleT}{Ejemplo}[chapter] % (Pedro: Example)
\theoremstyle{blacknumbox}
\newtheorem{vocabulary}{Vocabulario}[chapter]  % (Pedro: Vocabulary)
\newtheorem{definitionT}{Definición}[section]  % (Pedro: Definition)
\newtheorem{corollaryT}[dummy]{Corolario}  % (Pedro: Corollary)
\theoremstyle{ocrenum}
\newtheorem{proposition}[dummy]{Proposición} % (Pedro: Proposition)


\usepackage[framemethod=default]{mdframed}



\newcommand{\intoo}[2]{\mathopen{]}#1\,;#2\mathclose{[}}
\newcommand{\ud}{\mathop{\mathrm{{}d}}\mathopen{}}
\newcommand{\intff}[2]{\mathopen{[}#1\,;#2\mathclose{]}}
\newtheorem{notation}{Notation}[chapter]


\mdfdefinestyle{exampledefault}{%
rightline=true,innerleftmargin=10,innerrightmargin=10,
frametitlerule=true,frametitlerulecolor=green,
frametitlebackgroundcolor=yellow,
frametitlerulewidth=2pt}


% Theorem box
\newmdenv[skipabove=7pt,
skipbelow=7pt,
backgroundcolor=black!5,
linecolor=ocre,
innerleftmargin=5pt,
innerrightmargin=5pt,
innertopmargin=10pt,%5pt
leftmargin=0cm,
rightmargin=0cm,
innerbottommargin=5pt]{tBox}

% Exercise box	  
\newmdenv[skipabove=7pt,
skipbelow=7pt,
rightline=false,
leftline=true,
topline=false,
bottomline=false,
backgroundcolor=ocre!10,
linecolor=ocre,
innerleftmargin=5pt,
innerrightmargin=5pt,
innertopmargin=10pt,%5pt
innerbottommargin=5pt,
leftmargin=0cm,
rightmargin=0cm,
linewidth=4pt]{eBox}	

% Definition box
\newmdenv[skipabove=7pt,
skipbelow=7pt,
rightline=false,
leftline=true,
topline=false,
bottomline=false,
linecolor=ocre,
innerleftmargin=5pt,
innerrightmargin=5pt,
innertopmargin=10pt,%0pt
leftmargin=0cm,
rightmargin=0cm,
linewidth=4pt,
innerbottommargin=0pt]{dBox}	

% Corollary box
\newmdenv[skipabove=7pt,
skipbelow=7pt,
rightline=false,
leftline=true,
topline=false,
bottomline=false,
linecolor=gray,
backgroundcolor=black!5,
innerleftmargin=5pt,
innerrightmargin=5pt,
innertopmargin=10pt,%5pt
leftmargin=0cm,
rightmargin=0cm,
linewidth=4pt,
innerbottommargin=5pt]{cBox}

% Crea un entorno para cada tipo de theorem y le asigna un estilo 
% con ayuda de las cajas coloreadas anteriores
\newenvironment{theorem}{\begin{tBox}\begin{theoremeT}}{\end{theoremeT}\end{tBox}}
\newenvironment{exercise}{\begin{eBox}\begin{exerciseT}}{\hfill{\color{ocre}\tiny\ensuremath{\blacksquare}}\end{exerciseT}\end{eBox}}				  
\newenvironment{definition}{\begin{dBox}\begin{definitionT}}{\end{definitionT}\end{dBox}}	
\newenvironment{example}{\begin{exampleT}}{\hfill{\tiny\ensuremath{\blacksquare}}\end{exampleT}}		
\newenvironment{corollary}{\begin{cBox}\begin{corollaryT}}{\end{corollaryT}\end{cBox}}	

%	ENVIRONMENT remark
\newenvironment{remark}{\par\vspace{10pt}\small 
% Espacio blanco vertical sobre la nota y tamaño de fuente menor
\begin{list}{}{
\leftmargin=35pt % Indentación sobre la izquierda
\rightmargin=25pt}\item\ignorespaces % Indentación sobre la derecha
\makebox[-2.5pt]{\begin{tikzpicture}[overlay]
\node[draw=ocre!60,line width=1pt,circle,fill=ocre!25,font=\sffamily\bfseries,inner sep=2pt,outer sep=0pt] at (-15pt,0pt){\textcolor{ocre}{N}}; \end{tikzpicture}} % R naranja en un círculo (Pedro)
\advance\baselineskip -1pt}{\end{list}\vskip5pt} 
% Espaciado de línea más estrecho y espacio en blanco después del comentario


\newenvironment{solutionExe}{\par\vspace{10pt}\small 
\begin{list}{}{
\leftmargin=35pt 
\rightmargin=25pt}\item\ignorespaces 
\makebox[-2.5pt]{\begin{tikzpicture}[overlay]
\node[draw=ocre!60,line width=1pt,circle,fill=ocre!25,font=\sffamily\bfseries,inner sep=2pt,outer sep=0pt] at (-15pt,0pt){\textcolor{ocre}{S}}; \end{tikzpicture}} 
\advance\baselineskip -1pt}{\end{list}\vskip5pt} 

\newenvironment{solutionExa}{\par\vspace{10pt}\small 
\begin{list}{}{
\leftmargin=35pt 
\rightmargin=25pt}\item\ignorespaces 
\makebox[-2.5pt]{\begin{tikzpicture}[overlay]
\node[draw=ocre!60,line width=1pt,circle,fill=ocre!55,font=\sffamily\bfseries,inner sep=2pt,outer sep=0pt] at (-15pt,0pt){\textcolor{ocre}{S}}; \end{tikzpicture}} 
\advance\baselineskip -1pt}{\end{list}\vskip5pt} 

\usepackage{tcolorbox}

\usetikzlibrary{trees}

\theoremstyle{ocrenum}
\newtheorem{solutionT}[dummy]{Solución}  % (Pedro: Corollary)
\newenvironment{solution}{\begin{cBox}\begin{solutionT}}{\end{solutionT}\end{cBox}}	


\newcommand{\tcolorboxsolucion}[2]{%
\begin{tcolorbox}[colback=green!5!white,colframe=green!75!black,title=#1] 
 #2
 %\tcblower  % pone una línea discontinua
\end{tcolorbox}
}% final definición comando

\newtcbox{\mybox}[1][green]{on line,
arc=0pt,outer arc=0pt,colback=#1!10!white,colframe=#1!50!black, boxsep=0pt,left=1pt,right=1pt,top=2pt,bottom=2pt, boxrule=0pt,bottomrule=1pt,toprule=1pt}



\mdfdefinestyle{exampledefault}{%
rightline=true,innerleftmargin=10,innerrightmargin=10,
frametitlerule=true,frametitlerulecolor=green,
frametitlebackgroundcolor=yellow,
frametitlerulewidth=2pt}





\newcommand{\betheorem}{\begin{theorem}}
\newcommand{\eetheorem}{\end{theorem}}
\newcommand{\bedefinition}{\begin{definition}}
\newcommand{\eedefinition}{\end{definition}}

\newcommand{\beremark}{\begin{remark}}
\newcommand{\eeremark}{\end{remark}}
\newcommand{\beexercise}{\begin{exercise}}
\newcommand{\eeexercise}{\end{exercise}}
\newcommand{\beexample}{\begin{example}}
\newcommand{\eeexample}{\end{example}}
\newcommand{\becorollary}{\begin{corollary}}
\newcommand{\eecorollary}{\end{corollary}}


\newcommand{\besolutionExe}{\begin{solutionExe}}
\newcommand{\eesolutionExe}{\end{solutionExe}}
\newcommand{\besolutionExa}{\begin{solutionExa}}
\newcommand{\eesolutionExa}{\end{solutionExa}}


%%%%%%%%


% Caja Salida Markdown
\newmdenv[skipabove=7pt,
skipbelow=7pt,
rightline=false,
leftline=true,
topline=false,
bottomline=false,
backgroundcolor=GreenYellow!10,
linecolor=GreenYellow!80,
innerleftmargin=5pt,
innerrightmargin=5pt,
innertopmargin=10pt,%5pt
innerbottommargin=5pt,
leftmargin=0cm,
rightmargin=0cm,
linewidth=4pt]{mBox}	

%% RMarkdown
\newenvironment{markdownsal}{\begin{mBox}}{\end{mBox}}	

\newcommand{\bmarkdownsal}{\begin{markdownsal}}
\newcommand{\emarkdownsal}{\end{markdownsal}}


\usepackage{array}
\usepackage{multirow}
\usepackage{wrapfig}
\usepackage{colortbl}
\usepackage{pdflscape}
\usepackage{tabu}
\usepackage{threeparttable}
\usepackage{subfig} %new
%\usepackage{booktabs,dcolumn,rotating,thumbpdf,longtable}
\usepackage{dcolumn,rotating}  %new
\usepackage[graphicx]{realboxes} %new de: https://stackoverflow.com/questions/51633434/prevent-pagebreak-in-kableextra-landscape-table

%define el interlineado vertical
%\renewcommand{\baselinestretch}{1.5}

%define etiqueta para las Tablas o Cuadros
%\renewcommand\spanishtablename{Tabla}

%%\bibliographystyle{plain} %new no necesario


%%%%%%%%%%%% PARA USO CON biblatex
% \DefineBibliographyStrings{english}{%
%   backrefpage = {ver pag.\adddot},%
%   backrefpages = {ver pags.\adddot}%
% }

% \DefineBibliographyStrings{spanish}{%
%   backrefpage = {ver pag.\adddot},%
%   backrefpages = {ver pags.\adddot}%
% }
% 
% \DeclareFieldFormat{pagerefformat}{\mkbibparens{{\color{red}\mkbibemph{#1}}}}
% \renewbibmacro*{pageref}{%
%   \iflistundef{pageref}
%     {}
%     {\printtext[pagerefformat]{%
%        \ifnumgreater{\value{pageref}}{1}
%          {\bibstring{backrefpages}\ppspace}
%          {\bibstring{backrefpage}\ppspace}%
%        \printlist[pageref][-\value{listtotal}]{pageref}}}}
% 
%%% de kableExtra
\usepackage{booktabs}
\usepackage{longtable}
%\usepackage{array}
%\usepackage{multirow}
%\usepackage{wrapfig}
%\usepackage{float}
%\usepackage{colortbl}
%\usepackage{pdflscape}
%\usepackage{tabu}
%\usepackage{threeparttable}
\usepackage{threeparttablex}
\usepackage[normalem]{ulem}
\usepackage{makecell}
%\usepackage{xcolor}

%%%%%%% fin: latex_preambulo.tex PLLC


\begin{document}

\bibliographystyle{flexbib}



\raggedbottom

\ifdefined\ifprincipal
\else
\setlength{\parindent}{1em}
\pagestyle{fancy}
\setcounter{tocdepth}{4}
\tableofcontents

\fi

\ifdefined\ifdoblecara
\fancyhead{}{}
\fancyhead[LE,RO]{\scriptsize\rightmark}
\fancyfoot[LO,RE]{\scriptsize\slshape \leftmark}
\fancyfoot[C]{}
\fancyfoot[LE,RO]{\footnotesize\thepage}
\else
\fancyhead{}{}
\fancyhead[RO]{\scriptsize\rightmark}
\fancyfoot[LO]{\scriptsize\slshape \leftmark}
\fancyfoot[C]{}
\fancyfoot[RO]{\footnotesize\thepage}
\fi

\renewcommand{\headrulewidth}{0.4pt}
\renewcommand{\footrulewidth}{0.4pt}

\hypertarget{apuxe9ndice-tablas-de-ganancia-esperada}{%
\chapter{Apéndice: Tablas de ganancia
esperada}\label{apuxe9ndice-tablas-de-ganancia-esperada}}

\hypertarget{ganancias-esperadas-si-el-jugador-posee-una-mano-dura}{%
\section{Ganancias esperadas si el jugador posee una mano
dura}\label{ganancias-esperadas-si-el-jugador-posee-una-mano-dura}}

\begingroup\fontsize{12}{14}\selectfont

\begin{longtable}[t]{lcccccccccc}
\caption{\label{tab:unnamed-chunk-4}Tabla de ganancias si el jugador posee una mano dura}\\
\toprule
 & 2 & 3 & 4 & 5 & 6 & 7 & 8 & 9 & Figura & As\\
\midrule
4 & -0.11886 & -0.08037 & -0.05401 & -0.00814 & 0.00861 & -0.05625 & -0.13053 & -0.20904 & -0.30558 & -0.41769\\
5 & -0.13256 & -0.09301 & -0.06593 & -0.01959 & -0.00336 & -0.08642 & -0.15806 & -0.23391 & -0.32737 & -0.43593\\
6 & -0.14528 & -0.10475 & -0.07701 & -0.03022 & -0.01448 & -0.11709 & -0.18609 & -0.25927 & -0.34959 & -0.45454\\
7 & -0.11823 & -0.07720 & -0.05181 & -0.00503 & 0.02117 & -0.05942 & -0.19458 & -0.26254 & -0.34388 & -0.46623\\
8 & -0.04348 & -0.00540 & 0.01738 & 0.06347 & 0.09631 & 0.08249 & -0.06860 & -0.20400 & -0.28673 & -0.40266\\
\addlinespace
9 & 0.04602 & 0.07855 & 0.09855 & 0.14338 & 0.16961 & 0.16603 & 0.08359 & -0.06704 & -0.22003 & -0.32773\\
10 & 0.14494 & 0.17252 & 0.19027 & 0.23156 & 0.25304 & 0.24222 & 0.17702 & 0.09202 & -0.07622 & -0.23919\\
11 & 0.42572 & 0.45022 & 0.46595 & 0.49766 & 0.51427 & 0.48860 & 0.42334 & 0.35114 & 0.23152 & 0.04202\\
12 & -0.21183 & -0.18909 & -0.17448 & -0.14503 & -0.12961 & -0.15345 & -0.21405 & -0.28108 & -0.36780 & -0.46922\\
13 & -0.26813 & -0.24701 & -0.22100 & -0.16840 & -0.15900 & -0.21391 & -0.27018 & -0.33243 & -0.41295 & -0.50713\\
\addlinespace
14 & -0.31060 & -0.25740 & -0.22100 & -0.16840 & -0.15900 & -0.27006 & -0.32231 & -0.38012 & -0.45488 & -0.54234\\
15 & -0.31060 & -0.25740 & -0.22100 & -0.16840 & -0.15900 & -0.32220 & -0.37072 & -0.42439 & -0.49382 & -0.57503\\
16 & -0.31060 & -0.25740 & -0.22100 & -0.16840 & -0.15900 & -0.37062 & -0.41567 & -0.46551 & -0.52998 & -0.60538\\
17 & -0.17460 & -0.12280 & -0.09680 & -0.04570 & 0.00190 & -0.10540 & -0.38800 & -0.42540 & -0.46490 & -0.60860\\
18 & 0.10110 & 0.13990 & 0.15660 & 0.20420 & 0.26970 & 0.40280 & 0.10130 & -0.17980 & -0.23460 & -0.35050\\
\addlinespace
19 & 0.37820 & 0.39740 & 0.40590 & 0.44870 & 0.48290 & 0.61510 & 0.59150 & 0.29350 & -0.01420 & -0.10130\\
20 & 0.64140 & 0.64660 & 0.65010 & 0.68100 & 0.69660 & 0.76950 & 0.79150 & 0.76010 & 0.43770 & 0.15200\\
21 & 1.50000 & 1.50000 & 1.50000 & 1.50000 & 1.50000 & 1.50000 & 1.50000 & 1.50000 & 1.38630 & 1.03845\\
22 & -1.00000 & -1.00000 & -1.00000 & -1.00000 & -1.00000 & -1.00000 & -1.00000 & -1.00000 & -1.00000 & -1.00000\\
23 & -1.00000 & -1.00000 & -1.00000 & -1.00000 & -1.00000 & -1.00000 & -1.00000 & -1.00000 & -1.00000 & -1.00000\\
\addlinespace
24 & -1.00000 & -1.00000 & -1.00000 & -1.00000 & -1.00000 & -1.00000 & -1.00000 & -1.00000 & -1.00000 & -1.00000\\
25 & -1.00000 & -1.00000 & -1.00000 & -1.00000 & -1.00000 & -1.00000 & -1.00000 & -1.00000 & -1.00000 & -1.00000\\
26 & -1.00000 & -1.00000 & -1.00000 & -1.00000 & -1.00000 & -1.00000 & -1.00000 & -1.00000 & -1.00000 & -1.00000\\
27 & -1.00000 & -1.00000 & -1.00000 & -1.00000 & -1.00000 & -1.00000 & -1.00000 & -1.00000 & -1.00000 & -1.00000\\
28 & -1.00000 & -1.00000 & -1.00000 & -1.00000 & -1.00000 & -1.00000 & -1.00000 & -1.00000 & -1.00000 & -1.00000\\
\addlinespace
29 & -1.00000 & -1.00000 & -1.00000 & -1.00000 & -1.00000 & -1.00000 & -1.00000 & -1.00000 & -1.00000 & -1.00000\\
30 & -1.00000 & -1.00000 & -1.00000 & -1.00000 & -1.00000 & -1.00000 & -1.00000 & -1.00000 & -1.00000 & -1.00000\\
31 & -1.00000 & -1.00000 & -1.00000 & -1.00000 & -1.00000 & -1.00000 & -1.00000 & -1.00000 & -1.00000 & -1.00000\\
\bottomrule
\end{longtable}
\endgroup{}

\hypertarget{ganancias-esperadas-si-el-jugador-posee-una-mano-blanda}{%
\section{Ganancias esperadas si el jugador posee una mano
blanda}\label{ganancias-esperadas-si-el-jugador-posee-una-mano-blanda}}

\begin{Shaded}
\begin{Highlighting}[]
\NormalTok{estrategia\_G\_optima\_mano\_blanda }\OtherTok{\textless{}{-}} \FunctionTok{matrix}\NormalTok{(}\StringTok{""}\NormalTok{, }\AttributeTok{nrow =} \DecValTok{28}\NormalTok{, }\AttributeTok{ncol =} \DecValTok{10}\NormalTok{)}
\FunctionTok{rownames}\NormalTok{(estrategia\_G\_optima\_mano\_blanda) }\OtherTok{\textless{}{-}} \FunctionTok{as.character}\NormalTok{(}\DecValTok{4}\SpecialCharTok{:}\DecValTok{31}\NormalTok{)}
\FunctionTok{colnames}\NormalTok{(estrategia\_G\_optima\_mano\_blanda) }\OtherTok{\textless{}{-}}\NormalTok{ Cartas  }\CommentTok{\# del 2 al As}
\NormalTok{ganancia\_G\_optima\_mano\_blanda }\OtherTok{\textless{}{-}} \FunctionTok{matrix}\NormalTok{(}\ConstantTok{NA}\NormalTok{, }\AttributeTok{nrow =} \DecValTok{28}\NormalTok{, }\AttributeTok{ncol =} \DecValTok{10}\NormalTok{)}
\FunctionTok{rownames}\NormalTok{(ganancia\_G\_optima\_mano\_blanda) }\OtherTok{\textless{}{-}} \FunctionTok{as.character}\NormalTok{(}\DecValTok{4}\SpecialCharTok{:}\DecValTok{31}\NormalTok{)}
\FunctionTok{colnames}\NormalTok{(ganancia\_G\_optima\_mano\_blanda) }\OtherTok{\textless{}{-}}\NormalTok{ Cartas  }\CommentTok{\# del 2 al As}
\ControlFlowTok{for}\NormalTok{ (x }\ControlFlowTok{in} \DecValTok{22}\SpecialCharTok{:}\DecValTok{31}\NormalTok{) \{}
\NormalTok{  ganancia\_G\_optima\_mano\_blanda[}\FunctionTok{as.character}\NormalTok{(x),] }\OtherTok{\textless{}{-}} \SpecialCharTok{{-}}\DecValTok{1}
\NormalTok{  estrategia\_G\_optima\_mano\_blanda[}\FunctionTok{as.character}\NormalTok{(x),] }\OtherTok{\textless{}{-}} \StringTok{"Parar"}
\NormalTok{\}}

\ControlFlowTok{for}\NormalTok{ (x }\ControlFlowTok{in} \DecValTok{21}\SpecialCharTok{:}\DecValTok{4}\NormalTok{) \{}
  \ControlFlowTok{for}\NormalTok{ (b }\ControlFlowTok{in}\NormalTok{ Cartas) \{}
    
    \CommentTok{\# Verifica si es blackjack (21 con 2 cartas), solo posible si x == 21}
\NormalTok{    es\_blackjack }\OtherTok{\textless{}{-}}\NormalTok{ (x }\SpecialCharTok{==} \DecValTok{21}\NormalTok{)  }\CommentTok{\# Aquí podrías añadir verificación con número de cartas}
    
\NormalTok{    G0 }\OtherTok{\textless{}{-}} \FunctionTok{Calculo\_G\_0}\NormalTok{(x, b, resultados\_tabla,}\AttributeTok{es\_blackjack =}\NormalTok{ es\_blackjack)}
    
    \CommentTok{\# Esperanza de continuar}
\NormalTok{    G\_continuar }\OtherTok{\textless{}{-}} \DecValTok{0}
    \ControlFlowTok{for}\NormalTok{ (j }\ControlFlowTok{in} \DecValTok{1}\SpecialCharTok{:}\DecValTok{10}\NormalTok{) \{}
\NormalTok{      nueva\_x }\OtherTok{\textless{}{-}}\NormalTok{ x }\SpecialCharTok{+}\NormalTok{ Valor\_cartas\_mano\_blanda[j]}
      \ControlFlowTok{if}\NormalTok{ (nueva\_x }\SpecialCharTok{\textgreater{}} \DecValTok{21}\NormalTok{) \{}
\NormalTok{        nueva\_x\_dura }\OtherTok{\textless{}{-}}\NormalTok{ nueva\_x}\DecValTok{{-}10}
\NormalTok{        G\_continuar }\OtherTok{\textless{}{-}}\NormalTok{ G\_continuar }\SpecialCharTok{{-}}\NormalTok{ Probabilidades\_sacar\_carta[j]}\SpecialCharTok{*}\NormalTok{ganancia\_G\_optima[}\FunctionTok{as.character}\NormalTok{(nueva\_x\_dura),b]}
\NormalTok{      \} }\ControlFlowTok{else}\NormalTok{ \{}
\NormalTok{        G\_continuar }\OtherTok{\textless{}{-}}\NormalTok{ G\_continuar }\SpecialCharTok{+}\NormalTok{ Probabilidades\_sacar\_carta[j] }\SpecialCharTok{*}\NormalTok{ganancia\_G\_optima\_mano\_blanda[}\FunctionTok{as.character}\NormalTok{(nueva\_x),b] }
\NormalTok{      \}}
\NormalTok{    \}}
    
    \ControlFlowTok{if}\NormalTok{ (G0 }\SpecialCharTok{\textgreater{}=}\NormalTok{ G\_continuar) \{}
\NormalTok{      estrategia\_G\_optima\_mano\_blanda[}\FunctionTok{as.character}\NormalTok{(x), b] }\OtherTok{\textless{}{-}} \StringTok{"Parar"}
\NormalTok{      ganancia\_G\_optima\_mano\_blanda[}\FunctionTok{as.character}\NormalTok{(x),b] }\OtherTok{\textless{}{-}}\NormalTok{ G0}
\NormalTok{    \} }\ControlFlowTok{else}\NormalTok{ \{}
\NormalTok{      estrategia\_G\_optima\_mano\_blanda[}\FunctionTok{as.character}\NormalTok{(x), b] }\OtherTok{\textless{}{-}} \StringTok{"Continuar"}
\NormalTok{      ganancia\_G\_optima\_mano\_blanda[}\FunctionTok{as.character}\NormalTok{(x),b] }\OtherTok{\textless{}{-}}\NormalTok{ G\_continuar}
\NormalTok{    \}}
\NormalTok{  \}}
\NormalTok{\}}
\end{Highlighting}
\end{Shaded}

\begingroup\fontsize{12}{14}\selectfont

\begin{longtable}[t]{lcccccccccc}
\caption{\label{tab:unnamed-chunk-6}Tabla de ganancias si el jugador posee una mano blanda}\\
\toprule
 & 2 & 3 & 4 & 5 & 6 & 7 & 8 & 9 & Figura & As\\
\midrule
4 & 0.45469 & 0.43180 & 0.41540 & 0.40186 & 0.40762 & 0.49969 & 0.52246 & 0.53103 & 0.52637 & 0.52040\\
5 & 0.45018 & 0.42632 & 0.40928 & 0.39450 & 0.39979 & 0.49814 & 0.52142 & 0.53113 & 0.52846 & 0.52482\\
6 & 0.44184 & 0.41709 & 0.39973 & 0.38333 & 0.38644 & 0.48702 & 0.51950 & 0.52988 & 0.52841 & 0.52906\\
7 & 0.41662 & 0.39111 & 0.37407 & 0.35853 & 0.36054 & 0.45291 & 0.49629 & 0.51676 & 0.51662 & 0.52439\\
8 & 0.38475 & 0.36150 & 0.34594 & 0.34738 & 0.37050 & 0.47675 & 0.45299 & 0.47978 & 0.48779 & 0.49963\\
\addlinespace
9 & 0.42701 & 0.41705 & 0.40879 & 0.41539 & 0.43160 & 0.53662 & 0.54271 & 0.46875 & 0.45316 & 0.45944\\
10 & 0.54704 & 0.53490 & 0.52622 & 0.53040 & 0.54182 & 0.62808 & 0.64370 & 0.63136 & 0.53335 & 0.43981\\
11 & 0.72917 & 0.71554 & 0.70598 & 0.70179 & 0.70827 & 0.77418 & 0.78715 & 0.78588 & 0.74209 & 0.63904\\
12 & 0.41991 & 0.40097 & 0.38737 & 0.37396 & 0.37597 & 0.44560 & 0.47460 & 0.49210 & 0.49991 & 0.50694\\
13 & 0.42406 & 0.40367 & 0.38625 & 0.36523 & 0.36777 & 0.44582 & 0.47552 & 0.49477 & 0.50584 & 0.51681\\
\addlinespace
14 & 0.42557 & 0.39608 & 0.37516 & 0.35202 & 0.35375 & 0.44859 & 0.47881 & 0.49948 & 0.51331 & 0.52768\\
15 & 0.41831 & 0.38723 & 0.36524 & 0.34010 & 0.34119 & 0.45461 & 0.48431 & 0.50620 & 0.52243 & 0.53938\\
16 & 0.40495 & 0.37240 & 0.34993 & 0.32334 & 0.32230 & 0.44884 & 0.48946 & 0.51102 & 0.52818 & 0.55139\\
17 & 0.34836 & 0.31566 & 0.29578 & 0.26361 & 0.24368 & 0.32566 & 0.45589 & 0.48604 & 0.49866 & 0.54352\\
18 & 0.22655 & 0.19390 & 0.17485 & 0.20420 & 0.26970 & 0.40280 & 0.25198 & 0.38421 & 0.42315 & 0.46246\\
\addlinespace
19 & 0.37820 & 0.39740 & 0.40590 & 0.44870 & 0.48290 & 0.61510 & 0.59150 & 0.29350 & 0.30495 & 0.38076\\
20 & 0.64140 & 0.64660 & 0.65010 & 0.68100 & 0.69660 & 0.76950 & 0.79150 & 0.76010 & 0.43770 & 0.16254\\
21 & 1.50000 & 1.50000 & 1.50000 & 1.50000 & 1.50000 & 1.50000 & 1.50000 & 1.50000 & 1.38630 & 1.03845\\
22 & -1.00000 & -1.00000 & -1.00000 & -1.00000 & -1.00000 & -1.00000 & -1.00000 & -1.00000 & -1.00000 & -1.00000\\
23 & -1.00000 & -1.00000 & -1.00000 & -1.00000 & -1.00000 & -1.00000 & -1.00000 & -1.00000 & -1.00000 & -1.00000\\
\addlinespace
24 & -1.00000 & -1.00000 & -1.00000 & -1.00000 & -1.00000 & -1.00000 & -1.00000 & -1.00000 & -1.00000 & -1.00000\\
25 & -1.00000 & -1.00000 & -1.00000 & -1.00000 & -1.00000 & -1.00000 & -1.00000 & -1.00000 & -1.00000 & -1.00000\\
26 & -1.00000 & -1.00000 & -1.00000 & -1.00000 & -1.00000 & -1.00000 & -1.00000 & -1.00000 & -1.00000 & -1.00000\\
27 & -1.00000 & -1.00000 & -1.00000 & -1.00000 & -1.00000 & -1.00000 & -1.00000 & -1.00000 & -1.00000 & -1.00000\\
28 & -1.00000 & -1.00000 & -1.00000 & -1.00000 & -1.00000 & -1.00000 & -1.00000 & -1.00000 & -1.00000 & -1.00000\\
\addlinespace
29 & -1.00000 & -1.00000 & -1.00000 & -1.00000 & -1.00000 & -1.00000 & -1.00000 & -1.00000 & -1.00000 & -1.00000\\
30 & -1.00000 & -1.00000 & -1.00000 & -1.00000 & -1.00000 & -1.00000 & -1.00000 & -1.00000 & -1.00000 & -1.00000\\
31 & -1.00000 & -1.00000 & -1.00000 & -1.00000 & -1.00000 & -1.00000 & -1.00000 & -1.00000 & -1.00000 & -1.00000\\
\bottomrule
\end{longtable}
\endgroup{}

\hypertarget{ganancias-esperadas-en-caso-de-doblarse-o-no-doblarse}{%
\section{Ganancias esperadas en caso de doblarse o no
doblarse}\label{ganancias-esperadas-en-caso-de-doblarse-o-no-doblarse}}

\begingroup\fontsize{12}{14}\selectfont

\begin{longtable}[t]{lcccccccccc}
\caption{\label{tab:unnamed-chunk-8}Tabla de ganancias al doblarse o no hacerlo}\\
\toprule
 & 2 & 3 & 4 & 5 & 6 & 7 & 8 & 9 & Figura & As\\
\midrule
9 & 0.04602 & 0.10906 & 0.15500 & 0.24983 & 0.29935 & 0.16603 & 0.08359 & -0.06704 & -0.22003 & -0.32773\\
10 & 0.33860 & 0.39955 & 0.43818 & 0.52111 & 0.56122 & 0.38815 & 0.27846 & 0.14243 & -0.07622 & -0.23919\\
11 & 0.42572 & 0.45022 & 0.46595 & 0.49766 & 0.52663 & 0.48860 & 0.42334 & 0.35114 & 0.23152 & 0.04202\\
\bottomrule
\end{longtable}
\endgroup{}

\hypertarget{ganancias-esperadas-en-caso-de-abrirse-o-no-abrirse}{%
\section{Ganancias esperadas en caso de abrirse o no
abrirse}\label{ganancias-esperadas-en-caso-de-abrirse-o-no-abrirse}}

\begingroup\fontsize{12}{14}\selectfont

\begin{longtable}[t]{lcccccccccc}
\caption{\label{tab:unnamed-chunk-10}Tabla de ganancias al abrirse o no hacerlo}\\
\toprule
 & 2 & 3 & 4 & 5 & 6 & 7 & 8 & 9 & Figura & As\\
\midrule
2-2 & -0.11886 & -0.08037 & -0.05401 & 0.01916 & 0.05658 & -0.02577 & -0.13053 & -0.20904 & -0.30558 & -0.41769\\
3-3 & -0.14528 & -0.10475 & -0.07701 & -0.01628 & 0.01722 & -0.08658 & -0.18609 & -0.25927 & -0.34959 & -0.45454\\
4-4 & -0.04348 & -0.00540 & 0.01738 & 0.06347 & 0.09631 & 0.08249 & -0.06860 & -0.20400 & -0.28673 & -0.40266\\
5-5 & 0.14494 & 0.17252 & 0.19027 & 0.23156 & 0.25304 & 0.24222 & 0.17702 & 0.09202 & -0.07622 & -0.23919\\
6-6 & -0.21183 & -0.18909 & -0.16094 & -0.06669 & -0.03192 & -0.15345 & -0.21405 & -0.28108 & -0.36780 & -0.46922\\
\addlinespace
7-7 & -0.21421 & -0.13205 & -0.08220 & 0.01159 & 0.06902 & -0.06957 & -0.32231 & -0.38012 & -0.45488 & -0.54234\\
8-8 & -0.03586 & 0.03825 & 0.08204 & 0.17391 & 0.24082 & 0.23407 & -0.05906 & -0.35253 & -0.52998 & -0.60538\\
9-9 & 0.16841 & 0.23003 & 0.26784 & 0.35590 & 0.40746 & 0.41318 & 0.26172 & -0.03131 & -0.23460 & -0.35050\\
Figura-Figura & 0.64140 & 0.64660 & 0.65010 & 0.68100 & 0.69660 & 0.76950 & 0.79150 & 0.76010 & 0.43770 & 0.15200\\
As-As & 0.41991 & 0.40097 & 0.43818 & 0.52111 & 0.56122 & 0.44560 & 0.47460 & 0.49210 & 0.49991 & 0.50694\\
\bottomrule
\end{longtable}
\endgroup{}

\bibliography{bib/library.bib,bib/paquetes.bib}


%


\end{document}
