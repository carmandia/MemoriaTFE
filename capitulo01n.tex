\documentclass[12pt,a4paper,]{book}
\def\ifdoblecara{} %% set to true
\def\ifprincipal{} %% set to true
\let\ifprincipal\undefined %% set to false
\def\ifcitapandoc{} %% set to true
\let\ifcitapandoc\undefined %% set to false
\usepackage{lmodern}
% sin fontmathfamily
\usepackage{amssymb,amsmath}
\usepackage{ifxetex,ifluatex}
%\usepackage{fixltx2e} % provides \textsubscript %PLLC
\ifnum 0\ifxetex 1\fi\ifluatex 1\fi=0 % if pdftex
  \usepackage[T1]{fontenc}
  \usepackage[utf8]{inputenc}
\else % if luatex or xelatex
  \ifxetex
    \usepackage{mathspec}
  \else
    \usepackage{fontspec}
  \fi
  \defaultfontfeatures{Ligatures=TeX,Scale=MatchLowercase}
\fi
% use upquote if available, for straight quotes in verbatim environments
\IfFileExists{upquote.sty}{\usepackage{upquote}}{}
% use microtype if available
\IfFileExists{microtype.sty}{%
\usepackage{microtype}
\UseMicrotypeSet[protrusion]{basicmath} % disable protrusion for tt fonts
}{}
\usepackage[margin = 2.5cm]{geometry}
\usepackage{hyperref}
\hypersetup{unicode=true,
            pdfauthor={Nombre Completo Autor},
              pdfborder={0 0 0},
              breaklinks=true}
\urlstyle{same}  % don't use monospace font for urls
%
\usepackage[usenames,dvipsnames]{xcolor}  %new PLLC
\IfFileExists{parskip.sty}{%
\usepackage{parskip}
}{% else
\setlength{\parindent}{0pt}
\setlength{\parskip}{6pt plus 2pt minus 1pt}
}
\setlength{\emergencystretch}{3em}  % prevent overfull lines
\providecommand{\tightlist}{%
  \setlength{\itemsep}{0pt}\setlength{\parskip}{0pt}}
\setcounter{secnumdepth}{5}
% Redefines (sub)paragraphs to behave more like sections
\ifx\paragraph\undefined\else
\let\oldparagraph\paragraph
\renewcommand{\paragraph}[1]{\oldparagraph{#1}\mbox{}}
\fi
\ifx\subparagraph\undefined\else
\let\oldsubparagraph\subparagraph
\renewcommand{\subparagraph}[1]{\oldsubparagraph{#1}\mbox{}}
\fi

%%% Use protect on footnotes to avoid problems with footnotes in titles
\let\rmarkdownfootnote\footnote%
\def\footnote{\protect\rmarkdownfootnote}


  \title{}
    \author{Nombre Completo Autor}
      \date{18/11/2021}


%%%%%%% inicio: latex_preambulo.tex PLLC


%% UTILIZA CODIFICACIÓN UTF-8
%% MODIFICARLO CONVENIENTEMENTE PARA USARLO CON OTRAS CODIFICACIONES


%\usepackage[spanish,es-nodecimaldot,es-noshorthands]{babel}
\usepackage[spanish,es-nodecimaldot,es-noshorthands,es-tabla]{babel}
% Ver: es-tabla (en: https://osl.ugr.es/CTAN/macros/latex/contrib/babel-contrib/spanish/spanish.pdf)
% es-tabla (en: https://tex.stackexchange.com/questions/80443/change-the-word-table-in-table-captions)
\usepackage[spanish, plain, datebegin,sortcompress,nocomment,
noabstract]{flexbib}
 
\usepackage{float}
\usepackage{placeins}
\usepackage{fancyhdr}
% Solucion: ! LaTeX Error: Command \counterwithout already defined.
% https://tex.stackexchange.com/questions/425600/latex-error-command-counterwithout-already-defined
\let\counterwithout\relax
\let\counterwithin\relax
\usepackage{chngcntr}
%\usepackage{microtype}  %antes en template PLLC
\usepackage[utf8]{inputenc}
\usepackage[T1]{fontenc} % Usa codificación 8-bit que tiene 256 glyphs

%\usepackage[dvipsnames]{xcolor}
%\usepackage[usenames,dvipsnames]{xcolor}  %new
\usepackage{pdfpages}
%\usepackage{natbib}




% Para portada: latex_paginatitulo_mod_ST02.tex (inicio)
\usepackage{tikz}
\usepackage{epigraph}
\input{portadas/latex_paginatitulo_mod_ST02_add.sty}
% Para portada: latex_paginatitulo_mod_ST02.tex (fin)

% Para portada: latex_paginatitulo_mod_OV01.tex (inicio)
\usepackage{cpimod}
% Para portada: latex_paginatitulo_mod_OV01.tex (fin)

% Para portada: latex_paginatitulo_mod_OV03.tex (inicio)
\usepackage{KTHEEtitlepage}
% Para portada: latex_paginatitulo_mod_OV03.tex (fin)

\renewcommand{\contentsname}{Índice}
\renewcommand{\listfigurename}{Índice de figuras}
\renewcommand{\listtablename}{Índice de tablas}
\newcommand{\bcols}{}
\newcommand{\ecols}{}
\newcommand{\bcol}[1]{\begin{minipage}{#1\linewidth}}
\newcommand{\ecol}{\end{minipage}}
\newcommand{\balertblock}[1]{\begin{alertblock}{#1}}
\newcommand{\ealertblock}{\end{alertblock}}
\newcommand{\bitemize}{\begin{itemize}}
\newcommand{\eitemize}{\end{itemize}}
\newcommand{\benumerate}{\begin{enumerate}}
\newcommand{\eenumerate}{\end{enumerate}}
\newcommand{\saltopagina}{\newpage}
\newcommand{\bcenter}{\begin{center}}
\newcommand{\ecenter}{\end{center}}
\newcommand{\beproof}{\begin{proof}} %new
\newcommand{\eeproof}{\end{proof}} %new
%De: https://texblog.org/2007/11/07/headerfooter-in-latex-with-fancyhdr/
% \fancyhead
% E: Even page
% O: Odd page
% L: Left field
% C: Center field
% R: Right field
% H: Header
% F: Footer
%\fancyhead[CO,CE]{Resultados}

%OPCION 1
% \fancyhead[LE,RO]{\slshape \rightmark}
% \fancyhead[LO,RE]{\slshape \leftmark}
% \fancyfoot[C]{\thepage}
% \renewcommand{\headrulewidth}{0.4pt}
% \renewcommand{\footrulewidth}{0pt}

%OPCION 2
% \fancyhead[LE,RO]{\slshape \rightmark}
% \fancyfoot[LO,RE]{\slshape \leftmark}
% \fancyfoot[LE,RO]{\thepage}
% \renewcommand{\headrulewidth}{0.4pt}
% \renewcommand{\footrulewidth}{0.4pt}
%%%%%%%%%%
\usepackage{calc,amsfonts}
% Elimina la cabecera de páginas impares vacías al finalizar los capítulos
\usepackage{emptypage}
\makeatletter

%\definecolor{ocre}{RGB}{25,25,243} % Define el color azul (naranja) usado para resaltar algunas salidas
\definecolor{ocre}{RGB}{0,0,0} % Define el color a negro (aparece en los teoremas

%\usepackage{calc} 


%era if(csl-refs) con dolares
% metodobib: true


\usepackage{lipsum}

%\usepackage{tikz} % Requerido para dibujar formas personalizadas

%\usepackage{amsmath,amsthm,amssymb,amsfonts}
\usepackage{amsthm}


% Boxed/framed environments
\newtheoremstyle{ocrenumbox}% % Theorem style name
{0pt}% Space above
{0pt}% Space below
{\normalfont}% % Body font
{}% Indent amount
{\small\bf\sffamily\color{ocre}}% % Theorem head font
{\;}% Punctuation after theorem head
{0.25em}% Space after theorem head
{\small\sffamily\color{ocre}\thmname{#1}\nobreakspace\thmnumber{\@ifnotempty{#1}{}\@upn{#2}}% Theorem text (e.g. Theorem 2.1)
\thmnote{\nobreakspace\the\thm@notefont\sffamily\bfseries\color{black}---\nobreakspace#3.}} % Optional theorem note
\renewcommand{\qedsymbol}{$\blacksquare$}% Optional qed square

\newtheoremstyle{blacknumex}% Theorem style name
{5pt}% Space above
{5pt}% Space below
{\normalfont}% Body font
{} % Indent amount
{\small\bf\sffamily}% Theorem head font
{\;}% Punctuation after theorem head
{0.25em}% Space after theorem head
{\small\sffamily{\tiny\ensuremath{\blacksquare}}\nobreakspace\thmname{#1}\nobreakspace\thmnumber{\@ifnotempty{#1}{}\@upn{#2}}% Theorem text (e.g. Theorem 2.1)
\thmnote{\nobreakspace\the\thm@notefont\sffamily\bfseries---\nobreakspace#3.}}% Optional theorem note

\newtheoremstyle{blacknumbox} % Theorem style name
{0pt}% Space above
{0pt}% Space below
{\normalfont}% Body font
{}% Indent amount
{\small\bf\sffamily}% Theorem head font
{\;}% Punctuation after theorem head
{0.25em}% Space after theorem head
{\small\sffamily\thmname{#1}\nobreakspace\thmnumber{\@ifnotempty{#1}{}\@upn{#2}}% Theorem text (e.g. Theorem 2.1)
\thmnote{\nobreakspace\the\thm@notefont\sffamily\bfseries---\nobreakspace#3.}}% Optional theorem note

% Non-boxed/non-framed environments
\newtheoremstyle{ocrenum}% % Theorem style name
{5pt}% Space above
{5pt}% Space below
{\normalfont}% % Body font
{}% Indent amount
{\small\bf\sffamily\color{ocre}}% % Theorem head font
{\;}% Punctuation after theorem head
{0.25em}% Space after theorem head
{\small\sffamily\color{ocre}\thmname{#1}\nobreakspace\thmnumber{\@ifnotempty{#1}{}\@upn{#2}}% Theorem text (e.g. Theorem 2.1)
\thmnote{\nobreakspace\the\thm@notefont\sffamily\bfseries\color{black}---\nobreakspace#3.}} % Optional theorem note
\renewcommand{\qedsymbol}{$\blacksquare$}% Optional qed square
\makeatother



% Define el estilo texto theorem para cada tipo definido anteriormente
\newcounter{dummy} 
\numberwithin{dummy}{section}
\theoremstyle{ocrenumbox}
\newtheorem{theoremeT}[dummy]{Teorema}  % (Pedro: Theorem)
\newtheorem{problem}{Problema}[chapter]  % (Pedro: Problem)
\newtheorem{exerciseT}{Ejercicio}[chapter] % (Pedro: Exercise)
\theoremstyle{blacknumex}
\newtheorem{exampleT}{Ejemplo}[chapter] % (Pedro: Example)
\theoremstyle{blacknumbox}
\newtheorem{vocabulary}{Vocabulario}[chapter]  % (Pedro: Vocabulary)
\newtheorem{definitionT}{Definición}[section]  % (Pedro: Definition)
\newtheorem{corollaryT}[dummy]{Corolario}  % (Pedro: Corollary)
\theoremstyle{ocrenum}
\newtheorem{proposition}[dummy]{Proposición} % (Pedro: Proposition)


\usepackage[framemethod=default]{mdframed}



\newcommand{\intoo}[2]{\mathopen{]}#1\,;#2\mathclose{[}}
\newcommand{\ud}{\mathop{\mathrm{{}d}}\mathopen{}}
\newcommand{\intff}[2]{\mathopen{[}#1\,;#2\mathclose{]}}
\newtheorem{notation}{Notation}[chapter]


\mdfdefinestyle{exampledefault}{%
rightline=true,innerleftmargin=10,innerrightmargin=10,
frametitlerule=true,frametitlerulecolor=green,
frametitlebackgroundcolor=yellow,
frametitlerulewidth=2pt}


% Theorem box
\newmdenv[skipabove=7pt,
skipbelow=7pt,
backgroundcolor=black!5,
linecolor=ocre,
innerleftmargin=5pt,
innerrightmargin=5pt,
innertopmargin=10pt,%5pt
leftmargin=0cm,
rightmargin=0cm,
innerbottommargin=5pt]{tBox}

% Exercise box	  
\newmdenv[skipabove=7pt,
skipbelow=7pt,
rightline=false,
leftline=true,
topline=false,
bottomline=false,
backgroundcolor=ocre!10,
linecolor=ocre,
innerleftmargin=5pt,
innerrightmargin=5pt,
innertopmargin=10pt,%5pt
innerbottommargin=5pt,
leftmargin=0cm,
rightmargin=0cm,
linewidth=4pt]{eBox}	

% Definition box
\newmdenv[skipabove=7pt,
skipbelow=7pt,
rightline=false,
leftline=true,
topline=false,
bottomline=false,
linecolor=ocre,
innerleftmargin=5pt,
innerrightmargin=5pt,
innertopmargin=10pt,%0pt
leftmargin=0cm,
rightmargin=0cm,
linewidth=4pt,
innerbottommargin=0pt]{dBox}	

% Corollary box
\newmdenv[skipabove=7pt,
skipbelow=7pt,
rightline=false,
leftline=true,
topline=false,
bottomline=false,
linecolor=gray,
backgroundcolor=black!5,
innerleftmargin=5pt,
innerrightmargin=5pt,
innertopmargin=10pt,%5pt
leftmargin=0cm,
rightmargin=0cm,
linewidth=4pt,
innerbottommargin=5pt]{cBox}

% Crea un entorno para cada tipo de theorem y le asigna un estilo 
% con ayuda de las cajas coloreadas anteriores
\newenvironment{theorem}{\begin{tBox}\begin{theoremeT}}{\end{theoremeT}\end{tBox}}
\newenvironment{exercise}{\begin{eBox}\begin{exerciseT}}{\hfill{\color{ocre}\tiny\ensuremath{\blacksquare}}\end{exerciseT}\end{eBox}}				  
\newenvironment{definition}{\begin{dBox}\begin{definitionT}}{\end{definitionT}\end{dBox}}	
\newenvironment{example}{\begin{exampleT}}{\hfill{\tiny\ensuremath{\blacksquare}}\end{exampleT}}		
\newenvironment{corollary}{\begin{cBox}\begin{corollaryT}}{\end{corollaryT}\end{cBox}}	

%	ENVIRONMENT remark
\newenvironment{remark}{\par\vspace{10pt}\small 
% Espacio blanco vertical sobre la nota y tamaño de fuente menor
\begin{list}{}{
\leftmargin=35pt % Indentación sobre la izquierda
\rightmargin=25pt}\item\ignorespaces % Indentación sobre la derecha
\makebox[-2.5pt]{\begin{tikzpicture}[overlay]
\node[draw=ocre!60,line width=1pt,circle,fill=ocre!25,font=\sffamily\bfseries,inner sep=2pt,outer sep=0pt] at (-15pt,0pt){\textcolor{ocre}{N}}; \end{tikzpicture}} % R naranja en un círculo (Pedro)
\advance\baselineskip -1pt}{\end{list}\vskip5pt} 
% Espaciado de línea más estrecho y espacio en blanco después del comentario


\newenvironment{solutionExe}{\par\vspace{10pt}\small 
\begin{list}{}{
\leftmargin=35pt 
\rightmargin=25pt}\item\ignorespaces 
\makebox[-2.5pt]{\begin{tikzpicture}[overlay]
\node[draw=ocre!60,line width=1pt,circle,fill=ocre!25,font=\sffamily\bfseries,inner sep=2pt,outer sep=0pt] at (-15pt,0pt){\textcolor{ocre}{S}}; \end{tikzpicture}} 
\advance\baselineskip -1pt}{\end{list}\vskip5pt} 

\newenvironment{solutionExa}{\par\vspace{10pt}\small 
\begin{list}{}{
\leftmargin=35pt 
\rightmargin=25pt}\item\ignorespaces 
\makebox[-2.5pt]{\begin{tikzpicture}[overlay]
\node[draw=ocre!60,line width=1pt,circle,fill=ocre!55,font=\sffamily\bfseries,inner sep=2pt,outer sep=0pt] at (-15pt,0pt){\textcolor{ocre}{S}}; \end{tikzpicture}} 
\advance\baselineskip -1pt}{\end{list}\vskip5pt} 

\usepackage{tcolorbox}

\usetikzlibrary{trees}

\theoremstyle{ocrenum}
\newtheorem{solutionT}[dummy]{Solución}  % (Pedro: Corollary)
\newenvironment{solution}{\begin{cBox}\begin{solutionT}}{\end{solutionT}\end{cBox}}	


\newcommand{\tcolorboxsolucion}[2]{%
\begin{tcolorbox}[colback=green!5!white,colframe=green!75!black,title=#1] 
 #2
 %\tcblower  % pone una línea discontinua
\end{tcolorbox}
}% final definición comando

\newtcbox{\mybox}[1][green]{on line,
arc=0pt,outer arc=0pt,colback=#1!10!white,colframe=#1!50!black, boxsep=0pt,left=1pt,right=1pt,top=2pt,bottom=2pt, boxrule=0pt,bottomrule=1pt,toprule=1pt}



\mdfdefinestyle{exampledefault}{%
rightline=true,innerleftmargin=10,innerrightmargin=10,
frametitlerule=true,frametitlerulecolor=green,
frametitlebackgroundcolor=yellow,
frametitlerulewidth=2pt}





\newcommand{\betheorem}{\begin{theorem}}
\newcommand{\eetheorem}{\end{theorem}}
\newcommand{\bedefinition}{\begin{definition}}
\newcommand{\eedefinition}{\end{definition}}

\newcommand{\beremark}{\begin{remark}}
\newcommand{\eeremark}{\end{remark}}
\newcommand{\beexercise}{\begin{exercise}}
\newcommand{\eeexercise}{\end{exercise}}
\newcommand{\beexample}{\begin{example}}
\newcommand{\eeexample}{\end{example}}
\newcommand{\becorollary}{\begin{corollary}}
\newcommand{\eecorollary}{\end{corollary}}


\newcommand{\besolutionExe}{\begin{solutionExe}}
\newcommand{\eesolutionExe}{\end{solutionExe}}
\newcommand{\besolutionExa}{\begin{solutionExa}}
\newcommand{\eesolutionExa}{\end{solutionExa}}


%%%%%%%%


% Caja Salida Markdown
\newmdenv[skipabove=7pt,
skipbelow=7pt,
rightline=false,
leftline=true,
topline=false,
bottomline=false,
backgroundcolor=GreenYellow!10,
linecolor=GreenYellow!80,
innerleftmargin=5pt,
innerrightmargin=5pt,
innertopmargin=10pt,%5pt
innerbottommargin=5pt,
leftmargin=0cm,
rightmargin=0cm,
linewidth=4pt]{mBox}	

%% RMarkdown
\newenvironment{markdownsal}{\begin{mBox}}{\end{mBox}}	

\newcommand{\bmarkdownsal}{\begin{markdownsal}}
\newcommand{\emarkdownsal}{\end{markdownsal}}


\usepackage{array}
\usepackage{multirow}
\usepackage{wrapfig}
\usepackage{colortbl}
\usepackage{pdflscape}
\usepackage{tabu}
\usepackage{threeparttable}
\usepackage{subfig} %new
%\usepackage{booktabs,dcolumn,rotating,thumbpdf,longtable}
\usepackage{dcolumn,rotating}  %new
\usepackage[graphicx]{realboxes} %new de: https://stackoverflow.com/questions/51633434/prevent-pagebreak-in-kableextra-landscape-table

%define el interlineado vertical
%\renewcommand{\baselinestretch}{1.5}

%define etiqueta para las Tablas o Cuadros
%\renewcommand\spanishtablename{Tabla}

%%\bibliographystyle{plain} %new no necesario


%%%%%%%%%%%% PARA USO CON biblatex
% \DefineBibliographyStrings{english}{%
%   backrefpage = {ver pag.\adddot},%
%   backrefpages = {ver pags.\adddot}%
% }

% \DefineBibliographyStrings{spanish}{%
%   backrefpage = {ver pag.\adddot},%
%   backrefpages = {ver pags.\adddot}%
% }
% 
% \DeclareFieldFormat{pagerefformat}{\mkbibparens{{\color{red}\mkbibemph{#1}}}}
% \renewbibmacro*{pageref}{%
%   \iflistundef{pageref}
%     {}
%     {\printtext[pagerefformat]{%
%        \ifnumgreater{\value{pageref}}{1}
%          {\bibstring{backrefpages}\ppspace}
%          {\bibstring{backrefpage}\ppspace}%
%        \printlist[pageref][-\value{listtotal}]{pageref}}}}
% 
%%% de kableExtra
\usepackage{booktabs}
\usepackage{longtable}
%\usepackage{array}
%\usepackage{multirow}
%\usepackage{wrapfig}
%\usepackage{float}
%\usepackage{colortbl}
%\usepackage{pdflscape}
%\usepackage{tabu}
%\usepackage{threeparttable}
\usepackage{threeparttablex}
\usepackage[normalem]{ulem}
\usepackage{makecell}
%\usepackage{xcolor}

%%%%%%% fin: latex_preambulo.tex PLLC


\begin{document}

\bibliographystyle{flexbib}



\raggedbottom

\ifdefined\ifprincipal
\else
\setlength{\parindent}{1em}
\pagestyle{fancy}
\setcounter{tocdepth}{4}
\tableofcontents

\fi

\ifdefined\ifdoblecara
\fancyhead{}{}
\fancyhead[LE,RO]{\scriptsize\rightmark}
\fancyfoot[LO,RE]{\scriptsize\slshape \leftmark}
\fancyfoot[C]{}
\fancyfoot[LE,RO]{\footnotesize\thepage}
\else
\fancyhead{}{}
\fancyhead[RO]{\scriptsize\rightmark}
\fancyfoot[LO]{\scriptsize\slshape \leftmark}
\fancyfoot[C]{}
\fancyfoot[RO]{\footnotesize\thepage}
\fi

\renewcommand{\headrulewidth}{0.4pt}
\renewcommand{\footrulewidth}{0.4pt}

\hypertarget{Seccion1}{%
\chapter{Conocer la Teoría de Juegos}\label{Seccion1}}

\hypertarget{Seccion11}{%
\section{¿Qué es un juego?}\label{Seccion11}}

Cuando en la vida cotidiana llamamos juego a algo nos referimos a un
divertimento en el que una o varias personas participan (véase el
solitario, ajedrez o el poker). En estos juegos los participantes tienen
que cumplir una serie de reglas, y como resultado de sus decisiones
pueden ganar, pero también perder. En este proyecto nos centraremos en
los juegos con una o mas personas

En ellos los jugadores intentan maximizar sus resultados, es decir, en
el caso del poker ganar el mayor dinero posible, en ajedrez vencer al
rival lo mas rápido que puedan, etc. Todo esto los jugadores lo hacen
sabiendo que el resultado del juego depende no solo de ellos sino de lo
que hagan los demás jugadores. Pero estas situaciones no solo se da en
los juegos que conocemos como tal, sino en situaciones de la vida
diaria. Por ejemplo, cuando salimos del trabajo un viernes y cogemos el
coche para volver a casa, queremos hacerlo en el menor tiempo posible,
pero tenemos otros jugadores (no somos los únicos que queremos volver a
casa una vez terminado ese dia) y una serie de reglas, como no saltarnos
los limites de velocidad y respetar los semáforos.

Así pues, en adelante nos referiremos como juego a una situación en la
que varias personas interaccionan entre ellas y en la que el resultado
de cada uno no dependen solo de la estrategia que sigan ellos sino de la
de los demás jugadores.

A continuación definiremos una serie de términos que nos acompañarán a
lo largo del trabajo:

\begin{itemize}
\tightlist
\item
  \textbf{Jugadores}
\end{itemize}

Son los participantes del juego. Supondremos que actuan como seres
racionales

\begin{itemize}
\tightlist
\item
  \textbf{Reglas}
\end{itemize}

Son las condiciones en las que participan los jugadores. Podemos
diferenciar en:

\emph{Acciones de los jugadores} Son las decisiones que puede tomar cada
jugador en su turno de juego

\emph{Información} Conjunto de saberes que los jugadores tienen sobre
las acciones ya realizadas durante el juego

\emph{Estrategia} Definimos estrategia como el conjunto completo de
movimientos que tomaría el jugador en cada instante del juego.

\emph{Pagos} Utilidad o valoración que recibe cada jugador al terminar
el juego. Puede ser económica o no.

\hypertarget{Seccion12}{%
\section{Principio de racionalidad}\label{Seccion12}}

Al comienzo se ha comentado que suponíamos que los agentes o jugadores
actuaban de forma racional. En esta sección se estudiara que significa
que un jugador actúe de tal manera.

Partimos del supuesto de que los agentes o jugadores, ya sean personas,
empresas o gobiernos tienen deseos y preferencias de que tanto quieren
obtener del juego. Una vez que estos jugadores han establecido cuales
son sus preferencias, el principio de racionalidad establece que actuará
en función de las mismas, es decir estaríamos en todo momento buscando
nuestro máximo beneficio posible. De esta manera el agente actua
únicamente en función de sus preferencias y no se deja influir por la de
los demás.

Esto no significa que el agente siempre actue en contra de los demás
jugadores del juego necesariamente. Por ejemplo si su máxima preferencia
es el bienestar por igual para todos los jugadores esto no es negativo
para el resto. No obstante, el comportamiento usual es el egoísta, en el
que cada agente busca su máximo beneficio sin importarle las
consecuencias para los otros jugadores. A este comportamiento lo
llamaremos comúnmente auto-interesado que aunque no sean lo mismo en
muchas ocasiones llegan a coincidir

Las preferencias de cada agente son privadas y este las revela con sus
acciones y no previamente, para así evitar beneficiar a otros. Las
preferencias podemos establecerlas como relaciones binarias entre las
distintas alternativas o acciones que maneja cada jugador.

Con certidumbre, tendremos un conjunto de acciones
\(A=\{a_1, a_2,\cdots,a_n\}\), un conjunto de resultados que se derivan
de tales acciones \(R=\{r_1,r_2,\cdots,r_n\}\), y una función
\(x: A \rightarrow R\) estableciendo que a cada opción le corresponde un
único resultado (aplicación binaria) así tenemos una correspondencia
biunívoca entre ambas e identificamos la decisión con el resultado.
Cuando trabajamos en ambientes de incertidumbre las preferencias solo se
identifican con acciones puesto que no estamos seguros de que resultado
podemos obtener de ellas.

Definimos la relación binaria R de preferencia entre dos resultados
\(x_iRx_j\) para dos resultados cualesquiera \(x_i\), \(x_j\) (\(x_i\)
se prefiere a \(x_j\)), es decir que el resultado \(x_i\) es mejor o
igual que el resultado \(x_j\). A este tipo de preferencia se le conoce
como preferencia débil, mientras que si no solo se cumple esta relación,
sino que no se da la reciproca (no ocurre que \(x_jRx_i\)) entonces
hablamos de de preferencia estricta que se representa con una \(P\).
También podemos la relación de indiferencia si da igual preferir un
resultado que otro: \(x_iIx_j\) si y solo si \(x_iRx_j\) y \(x_jRx_i\).

Por lo tanto consideramos que el agente es racional cuando actúa en
función de sus preferencias y estas tienen una jerarquía interna. Estas
deben cumplir las siguientes propiedades:

\begin{enumerate}
\def\labelenumi{\arabic{enumi}.}
\tightlist
\item
  Completitud: \[
  \forall x_i,x_j, \ \ \ x_iRx_j \ ó \ x_jRx_i \ ó \ (x_iRx_j \ y \ x_jRx_i )
  \]
\item
  Reflexividad: \[
  \forall x_i, \ x_iRx_i
  \]
\item
  Transitividad: \[
  \forall x_i,x_j,x_k \ \ \text{tenemos que si:} \ \ x_iRx_j \ y \ x_jRx_k \ \ \rightarrow x_iRx_k
  \] Con esta propiedad evitamos la inconsistencia de las elecciones
\end{enumerate}

\hypertarget{Seccion13}{%
\section{Utilidad}\label{Seccion13}}

Ya tenemos establecidas las relaciones de preferencia entre las
distintas alternativas de manera racional como hemos explicado en el
apartado anterior. Una vez tenemos esto, para simplificar las
operaciones, traducimos estas preferencias a un orden cuantitativo
mediante una función \(U \ : \ X \rightarrow \mathbb{R}\) que le asigna
un valor real a cada una de las preferencias. Así, en vez de hablar por
ejemplo de \(\exists \ x_i / \ \forall x_j \in X, \ x_j \neq x_i\) se
tiene que: \(x_iRx_j\) hablamos de manera cuantitativa como
\(\exists x_i / \ \forall x_j \in X, x_j \neq x_i, \ U(x_i)\geq U(x_j)\)
Este hecho (de que se cumpla que la utilidad de la alternativa preferida
es mayor que la del resto de alternativas) solo lo tenemos en el caso de
que trabajemos en un ámbito de certidumbre:

\textbf{Ejemplo} Imaginemos que nos encontramos con 3 alternativas:
\(X=\{ \text{coche gratis},\text{2 semanas de vacaciones pagadas}, \text{10.000€ en mano} \}\),
para simplificar, \(X=\{x_1,x_2,x_3 \}\) con las siguientes
preferencias: \(x_2Rx_3, \ x_2Rx_1 \ y \ x_1Rx_3\) les asignamos un
valor o utilidad a cada una de estas alternativas (una utilidad) que
exprese de manera numérica estas preferencias:
\(U \ : \ X \rightarrow \mathbb{R}\) tal que
\(U(x_2) = 10, \ U(x_1)=5 \ y \ U(x_3) = 0\)

Si nos encontramos en un ámbito de incertidumbre no podemos asegurarnos
de que una vez establecidas las preferencias, estas tengan una utilidad
que exprese esa preferencia. En este caso nos encontramos una serie de
estados de la naturaleza \(E=\{e_1, \cdots, e_p \}\) tal que en función
de la alternativa que decidamos \(A=\{a_1, \cdots, a_n\}\) tendremos
unos resultados \(X=\{x_1, \cdots, x_n \}\) que dependen de ambos.

\textbf{Ejemplo} Tenemos los siguientes estados de la naturaleza
\(E=\{e_1=\text{Sequía}, e_2= \text{Lluvia} \}\) las alternativas
\(A=\{a_1=\text{Recolectar ahora},a_2= \text{Recolectar en un mes} \}\)
y se pueden producir los siguientes resultados
\(X= \{x_1=\text{Ganancias},x_2=\text{Pérdidas} \}\) que siguen la
siguiente relación: \[
\begin{array}{c| c c}
 & Sequía & Lluvia \\
 & e_1 & e_2 \\
\hline
a_1 & Ganancias & Pérdidas \\
\hline
a_2 & Pérdidas & Ganancias \\
\hline
\end{array}
\] Así pues el resultado que obtendremos dependerá de la decisión que
tomemos y el estado de la naturaleza que se presente. Por tanto, para
calcular la utilidad esperada de una acción cualquiera tendremos que
calcular:

\(Sea \ a\) una acción cualquiera,
\[U(a) = \sum_{e \in E} p(e)U(x(a,e)), \ \sum_{e \in E} p(e)=1\] donde e
son los estados de la naturaleza y \(U(x(a,e))\) es la utilidad de
elegir la alternativa a cuando ocurre el estado e de la naturaleza.

\hypertarget{Seccion14}{%
\section{Utilidad de Von Neumann-Morgenstern y actitudes ante el
riesgo}\label{Seccion14}}

Venimos buscando asociar a cada alternativa un orden de preferencia. Las
funciones de utilidad de Von Neumann-Morgenstern tenemos un metodo no
arbitrario para asignar valores numéricos a los resultados. Para
explicar esto comentamos una serie de definiciones básicas en este
aspecto

\textbf{Definición}

Una lotería simple en \(X\) es una distribución de probabilidad en
\(X\). Es decir, se dice que \(L\) es una lotería simple en \(X\) si:

\(L= \{(p_1,p_2, \cdots,p_n) \in \mathbb{R}^n: \ p_i \geq 0, \ para \ cada \ i=1,2,\cdots,n, \ y \ \sum_{i=1,\cdots,n} p_i=0 \}\),
donde \(p_i\) es la probabilidad de que ocurra la alternativa
\(a_i, \ i=1,\cdots,n\)

A partir de esta definición podemos definir el siguiente conjunto:

\[L_A = \{(p_1,\cdots,p_n) \in \mathbb{R}^n: p_i \geq 0, i=1,\cdots,n
\ y \ \sum_{i=1,\cdots,n} p_i=1 \}\] Conjunto de todas las loterías
simples sobre un conjunto de alternativas A.

Ahora ya estamos en condiciones de definir las funciones de utilidad de
Von Neumann-Morgenstern:

\textbf{Definición}

Así pues, una función \(U: \ L_A \rightarrow \mathbb{R}\) es una función
de utilidad esperada de Von Neumann-Morgenstern (VN-M) si existen \(n\)
números \(u_1, \cdots,u_n\), asociados a \(a_1, \cdots, a_n\) tales que
para cada lotería \(L \in L_A\) se verifica:
\(U(L) \ = \ u_1p_1 \ + u_2p_2 \ + \ \cdots \ u_np_n\)

A continuación enunciamos el Teorema de utilidad de Von
Neumann-Morgenstern:

\textbf{Teorema}

Supongamos la relación de preferencia \(R\) sobre \(L_A\) en las
condiciones estudiadas. Entonces R admite una representación en forma de
utilidad esperada de Von Neumann-Morgenstern, es decir, existen
\(u(a_1), \cdots, u(a_n)\), tales que\\
\[\forall L,L' \in L_A, \ L=(p_1,\cdots,p_n) \ y \ L'=(p_1', \cdots, p_n') \iff \sum_{i=1, \cdots, n} p_iu(a_i) \geq \sum_{i=1, \cdots, n} p_i'u(x_i)\]

\hypertarget{Seccion141}{%
\subsection{Actitudes ante el riesgo}\label{Seccion141}}

En este Apartado trabajaremos suponiendo que \(A=\mathbb{R}\)

Decimos que un agente es si el valor esperado de cualquier lotería \(L\)
es tan preferida o mas que dicha lotería. Si ocurre al contrario, que la
lotería sea igual o mas preferida que su valor esperado decimos que el
agente es propenso al riesgo. Si tenemos una situación de indiferencia,
decimos que el agente es neutral.

En términos de la función de utilidad \(u\) tenemos el siguiente
teorema:

Sea una función de utilidad \(u: [a,b] \ \rightarrow \ \mathbb{R}\) que
es estrictamente creciente y de clase \(C^2(\mathbb{R})\) entonces:

\begin{enumerate}
\def\labelenumi{\arabic{enumi}.}
\item
  Si \(u^{''}(x) \leq 0, \ \forall \ x \in \ [a,b]\), es decir, \(u\) es
  cóncava, entonces el jugador es conservador.
\item
  Si \(u^{''}(x) \geq 0, \ \forall \ x \in \ [a,b]\), es decir, \(u\) es
  convexa, entonces el jugador es arriesgado.
\item
  Si \(u^{''}(x) = 0, \ \forall \ x \in \ [a,b]\), es decir, \(u\) es
  lineal, entonces el jugador es indiferente al riesgo.
\end{enumerate}

\bibliography{bib/library.bib,bib/paquetes.bib}


%


\end{document}
